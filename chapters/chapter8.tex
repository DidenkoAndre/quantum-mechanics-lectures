\chapter{Угловой момент}

\section{Повороты и оператор углового момента. Изотропность пространства и сохренение глового момента в квантовой механике.}

[картинка]
(поворот можно осуществлять путём подбора единичного вектора $\vec{n}$ и угла поворота $\rchi$)

$$
\ket{\psi;2} = \op{R}(\vec{\rchi}) \ket{\psi;1}
$$

$\op{R}(\vec{\rchi})$ -- оператор поворота.

Примем, что $\bk{\psi;2}{\psi;2} = \bk{\psi;1}{\psi;1}$, тогда $\op{R}^\dag \op{R} = \mathds{1}$, т.е. $\op{R}$ -- унитарный оператор.

Введём $\op{R}$ по аналогии с оператором эволюции (см. \eqref{eq:6_4_4}):
\begin{equation}
\label{eq:8_1_1}
\boxed {
	\op{R}(\op{\rchi}) \equiv e^{-(i/\hbar) \op{\vec{J}} \vec{\rchi}}
}
\end{equation}
где $\op{\vec{J}}$ -- некоторый векторный эрмитовый оператор, не зависящий от времени.

Из изотропности пространства следует, что оба состояния удовлятворяют уравнению Шрёдингера:
$$
i\hbar \pd{\ket{\psi;2}}{t} = i\hbar \op{R}(\vec{\rchi}) \pd{\ket{\psi;1}}{t} = \op{R}(\vec{\rchi}) \op{H}\ket{\psi;1}
$$

$$
i\hbar \pd{\ket{\psi;2}}{t} = \op{H}\ket{\psi;2} = \op{H} \op{R}(\vec{\rchi}) \ket{\psi;1}
$$

Сравнивая правые части, легко видеть, что $\brs{\op{H}, \op{R}(\vec{\rchi})} = 0$

Распишем экспоненту в \eqref{eq:8_1_1} в виде ряда:
$$
\op{R}(\vec{\rchi}) = \sum_{k=0}^{\infty} \frac{1}{k!} \brc{- \frac{i}{\hbar} \op{\vec{J}} \vec{\rchi}}^k
$$

Подставляя её вусловие коммутации, получим:
\begin{equation}
\label{eq:8_1_2}
\boxed {
	\brs{\op{H}, \op{\vec{J}}} = 0
}
\end{equation}

\begin{excr}
доказать \eqref{eq:8_1_2}
\end{excr}

$\vec{J}$ -- интеграл движения (см. \S 9 т.1 Л-Л <<Сохранение углового момента>>)

$\op{\vec{J}}$ -- \textbf{оператор углового момента} ($\op{\vec{J}} = \brcr{\op{L}, \op{S}, \op{L}+\op{S}}$, где $\op{L}$ -- оператор орбитального момента, $\op{S}$ --оператор спинового момента, $(\op{L} + \op{S})$ -- полный момент)


\section{Коммутационные соотношения для оператора углового момента. Система собственных векторов операторов $\op{\vec{j}}^2$ и $\op{\vec{j}_z}$}


\section{Спин частицы. Матрицы Паули.}


\section{Оператор орбитального момента частицы в координатном представлении (декартовы и сферические координаты).}


\section{Сферические гармоники}