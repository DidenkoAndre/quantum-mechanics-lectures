\chapter{Угловой момент}

\section{Повороты и оператор углового момента. Изотропность пространства и сохренение глового момента в квантовой механике.}

[картинка]
(поворот можно осуществлять путём подбора единичного вектора $\vec{n}$ и угла поворота $\rchi$)

$$
\ket{\psi;2} = \op{R}(\vec{\rchi}) \ket{\psi;1}
$$

$\op{R}(\vec{\rchi})$ -- оператор поворота.

Примем, что $\bk{\psi;2}{\psi;2} = \bk{\psi;1}{\psi;1}$, тогда $\op{R}^\dag \op{R} = \mathds{1}$, т.е. $\op{R}$ -- унитарный оператор.

Введём $\op{R}$ по аналогии с оператором эволюции (см. \eqref{eq:6_4_4}):
\begin{equation}
\label{eq:8_1_1}
\boxed {
	\op{R}(\op{\rchi}) \equiv e^{-(i/\hbar) \op{\vec{J}} \vec{\rchi}}
}
\end{equation}
где $\op{\vec{J}}$ -- некоторый векторный эрмитовый оператор, не зависящий от времени.

Из изотропности пространства следует, что оба состояния удовлятворяют уравнению Шрёдингера:
$$
i\hbar \pd{\ket{\psi;2}}{t} = i\hbar \op{R}(\vec{\rchi}) \pd{\ket{\psi;1}}{t} = \op{R}(\vec{\rchi}) \op{H}\ket{\psi;1}
$$

$$
i\hbar \pd{\ket{\psi;2}}{t} = \op{H}\ket{\psi;2} = \op{H} \op{R}(\vec{\rchi}) \ket{\psi;1}
$$

Сравнивая правые части, легко видеть, что $\brs{\op{H}, \op{R}(\vec{\rchi})} = 0$

Распишем экспоненту в \eqref{eq:8_1_1} в виде ряда:
$$
\op{R}(\vec{\rchi}) = \sum_{k=0}^{\infty} \frac{1}{k!} \brc{- \frac{i}{\hbar} \op{\vec{J}} \vec{\rchi}}^k
$$

Подставляя её вусловие коммутации, получим:
\begin{equation}
\label{eq:8_1_2}
\boxed {
	\brs{\op{H}, \op{\vec{J}}} = 0
}
\end{equation}

\begin{excr}
доказать \eqref{eq:8_1_2}
\end{excr}

$\vec{J}$ -- интеграл движения (см. \S 9 т.1 Л-Л <<Сохранение углового момента>>)

$\op{\vec{J}}$ -- \textbf{оператор углового момента} ($\op{\vec{J}} = \brcr{\op{L}, \op{S}, \op{L}+\op{S}}$, где $\op{L}$ -- оператор орбитального момента, $\op{S}$ --оператор спинового момента, $(\op{L} + \op{S})$ -- полный момент)


\section{Коммутационные соотношения для оператора углового момента. Система собственных векторов операторов $\op{\vec{j}}^2$ и $\op{\vec{j}_z}$}

Обозначим: $\boxed{\op{\vec{j}} = \frac{\op{\vec{J}}}{\hbar}}$

\begin{defn}
векторный оператор $\op{\vec{j}} = \brcr{\op{j_x}, \op{j_y}, \op{j_z}}$ называют \textbf{оператором углового момента}, если все его компоненты являются \textbf{наблюдаемыми} (эрмитовыми) и удовлетворяют коммутационным соотношениям:
\begin{equation}
\label{eq:8_2_1}
\boxed {
	\brs{\op{j_i}, \op{j_k}} = i e_{ikl} \op{j_l}
}
\end{equation}
(где $e_{ikl}$ -- антисимметричный тензор)
\end{defn}

$$
\op{\vec{j}}^2 = \op{j_x}^2 + \op{j_y}^2 + \op{j_z}^2
$$

Совместная измеримость $\op{j}^2$ возможна только с одной компонентой:
\begin{equation}
\label{eq:8_2_2}
\brs{\op{j}^2, \op{j_i}} = 0
\end{equation}

\begin{excr}
Доказать \eqref{eq:8_2_2} с помощью \eqref{eq:8_2_1}
\end{excr}

$\ket{jm}$:
\begin{equation}
\label{eq:8_2_3}
\left\{
\begin{aligned}
\op{\vec{j}}^2 \ket{jm} = \lambda(j) \ket{jm} \\
\op{\vec{j}}_z \ket{jm} = m \ket{jm}
\end{aligned}
\right.
\end{equation}

Условие ортонормировки:
$$
\bk{jm}{j'm'} = \delta_{jj'} \delta{mm'}
$$

$$
\avg{\op{\vec{j}}^2} = \avg{\op{j_x}^2} + \avg{\op{j_y}^2} + \avg{\op{j_z}^2} \geqslant \avg{\op{j_z}^2}
$$

То есть:
$$
\lambda(j) \geqslant m^2 ~~\rightarrow~~ \left\{
\begin{aligned}
m_{min} &\leqslant m \leqslant m_{max}\\
m_{max} &= - m_{min}
\end{aligned}
\right.
$$

$m_{max} \equiv j$, тогда $m_{min} = -j$

Обозначим:
\begin{equation}
\label{eq:8_2_4}
\left\{
\begin{aligned}
\op{j}_{+} &= \op{j}_{x} + i \op{j}_{y} \\
\op{j}_{-} &= \op{j}_{x} - i \op{j}_{y} = \brc{\op{j}_{+}}^\dag
\end{aligned}
\right.
\end{equation}

\begin{equation}
\label{eq:8_2_5}
\brs{\op{j_z}, \op{j}_{\pm}} = \pm \op{j_{\pm}}
\end{equation}

\begin{excr}
Доказать \eqref{eq:8_2_5} с использованием \eqref{eq:8_2_4} и \eqref{eq:8_2_1}
\end{excr}

Из \eqref{eq:8_2_5}:
$$
\op{j_z} \brc{\op{j}_{\pm} \ket{jm}} = \brc{\op{j}_{\pm} \op{j_z} \pm \op{j}_{\pm}} \ket{jm} = (m \pm 1) \brc{\op{j}_{\pm} \ket{jm}}
$$

Будем называть $\op{j}_{+}$ \textbf{оператором повышения}, а $\op{j}_{-}$ -- \textbf{оператором понижения}

\begin{equation}
\label{eq:8_2_6}
\left\{
\begin{aligned}
\op{j}_{+} \ket{j, m-1} &= \alpha_m \ket{j, m}\\
\op{j}_{-} \ket{j, m} &= \beta_m \ket{j, m-1}\\
\end{aligned}
\right.
\end{equation}

Заметим, что:
$$
\begin{gathered}
\op{j}_{+} \ket{jj} = 0,~~\text{или}~~ \alpha_{j+1} = 0\\
\op{j}_{-} \ket{j, -j} = 0,~~\text{или}~~ \beta_{-j} = 0
\end{gathered}
$$

Цепочка понижения:
$$
\left\downarrow
\begin{aligned}
\op{j}_{-} \ket{jj} & \sim \ket{j, j-1}\\
(\op{j}_{-})^2 \ket{jj} & \sim \ket{j, j-2}\\
\cdots \\
(\op{j}_{-})^N \ket{jj} & \sim \ket{j, j-N},~~~N \in \mathbb{N} \cup \brcr{0} \\
\end{aligned}
\right.
$$

$j - N = -j$, т.е. $j = \frac{N}{2}$, следовательно $j$ принимает только целые и полуцелые значения:
$$
j = 0, \frac{1}{2}, 1, \frac{3}{2} \cdots
$$

Если $j$ фиксировано:
$$
\underbrace{m = -j, -j + 1, \cdots, j}_{(2j + 1)~\text{значения}}
$$

\noindent
$j$ -- квантовое число момента количества движения частицы\\
$m$ -- матричное квантовое число

Значение $\lambda(j)$ пока неизвестно. При его определении будем считать, что $m_{max} = m_{min} = j$.

$$
\op{j}_{-} \op{j}_{+} \ket{jj} = 0
$$
$$
\op{j}_{-} \op{j}_{+} = \op{\vec{j}}^2 - \op{j_z}^2 - \op{j_z}
$$
\begin{excr}
доказать предыдущее равенство используя \eqref{eq:8_2_4} и \eqref{eq:8_2_1}
\end{excr}

$$
\brc{\op{\vec{j}}^2 - \op{j_z}^2 - \op{j_z}} \ket{jj} = 0 ~~\rightarrow~~ \lambda(j) = j(j+1)
$$

\begin{equation}
\label{eq:8_2_7}
\boxed {
	\op{\vec{j}}^2 \ket{jm} = j(j+1) \ket{jm}
}
\end{equation}

Из \eqref{eq:8_2_6}:
$$
\alpha_m = \bfk{jm}{\op{j}_{+}}{j, m-1} = \bk{\op{j}_{-}jm}{j, m-1} = \left. \bfk{j,m-1}{\op{j}_{-}}{jm}^* \right|_{\text{\eqref{eq:8_2_6}}} = \beta_m^*
$$

Теперь необходимо найти $\beta_{-j+1}, \beta_{-j+2}, \cdots, \beta_{j}$

$$
\left. \op{j}_{+} \op{j}_{-} \ket{jm} \right|_{\text{\eqref{eq:8_2_6}}} = \beta_m \op{j}_{+} \ket{j, m-1} = \abs{\beta_m}^2 \ket{jm}
$$

С другой стороны:
$$
\op{j}_{+} \op{j}_{-} = \op{j}^2 - \op{j_z}^2 + \op{j_z}
$$

\begin{excr}
доказать предыдущее равенство используя \eqref{eq:8_2_4} и \eqref{eq:8_2_1}
\end{excr}

$$
\op{j}_{+} \op{j}_{-} \ket{jm} = \left. (\op{j}^2 - \op{j_z}^2 + \op{j_z}) \ket{jm} \right|_{\text{\eqref{eq:8_2_7}, \eqref{eq:8_2_3}}} = \brc{j(j+1) - m^2 + m} \ket{jm}
$$

Фазу $\ket{jm}$ можно подобрать так, чтобы $\alpha_m = \beta_m  = \abs{\beta_m}$

$$
\beta_m = \sqrt{j^2 + j - m^2 + m} = \sqrt{(j+m)(j-m+1)} = \alpha_m
$$

$$
\begin{gathered}
\op{j}_{-} \ket{jm} = \sqrt{(j+m)(j-m+1)} \ket{j, m-1} \\
\op{j}_{+} \ket{jm} = \beta_{m+1} \ket{j, m+1} = \sqrt{(j-m)(j+m+1)} \ket{j, m+1}
\end{gathered}
$$

\begin{equation}
\label{eq:8_2_8}
\boxed {
	(\op{j_x} \pm i \op{j_y}) \ket{jm} = \sqrt{(j \mp m)(j \pm m + 1)} \ket{j, m \pm 1}
}
\end{equation}


\section{Спин частицы. Матрицы Паули.}


\section{Оператор орбитального момента частицы в координатном представлении (декартовы и сферические координаты).}


\section{Сферические гармоники}
