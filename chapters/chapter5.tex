\chapter{Квантовая динамика частицы}

\section{Уравнение непрерывности для плотности вероятности. Плотность тока вероятности. Коэффициенты прохождения и отражения.}

Гамильтониан частицы, движущейся в стационарном потенциальном поле $U(\vr)$, имеет вид:

$$
  \left. \op{H}=\frac{\op{\vp^2}}{2m}+U(\vr) \right|_{\op{\vp} = 
  -i\hbar\vec{\nabla}} =-\frac{\hbar^2}{2m} \Delta + U(\vr)
$$%
%
Исходим из того, что уравнение Шрёдингера, определяющее волновую функцию частицы, установлено:
$$
  i\hbar\pd{}{t} \Psi(\vr,t) = \op{H}\Psi(\vr,t)
$$%
%
Временное (нестационарное) уравнение Шрёдингера \eqref{eq:1_3_1} и \eqref{eq:1_3_2} фактически вводилось как (четвёртый) постулат квантовой механики.

Домножая уравнение Шрёдингера на $\Psi^*$, а комплексно сопряжённое уравнение Шрёдингера на $\Psi$, получаем:

$$
i\hbar\Psi^*\pd{}{t}\Psi=\Psi^*\op{H}\Psi ~~~~~~~~~ -i\hbar\Psi\pd{}{t}\Psi^*=\Psi(\op{H}\Psi)^*
$$%
%
Вычитая из первого уравнение второе, находим:

\begin{equation}
  \label{eq:5_1_1}
  i \hbar \brc{\Psi^*\pd{\Psi}{t}+\Psi\pd{\Psi^*}{t}} = 
    \Psi^*(\op{H}\Psi)-\Psi(\op{H}\Psi)^*
\end{equation}

В левой части этого уравнения имеем:

$$
  \left.
    \pd{}{t}\Psi^*(\vr,t)\Psi(\vr,t)
  \right|_{\rho(\vr,t)=|\Psi(\vr,t)|^2} = \pd{\rho}{t}
$$%
%
где $\rho(\vr,t)$ -- плотность вероятности обнаружить частицу в момент времени $t$ в точке $\vr$ (см. соотношения \eqref{eq:1_4_1} и \eqref{eq:1_4_2}). Правая часть преобразуется к виду

$$
\begin{gathered}
  \left.
    -\frac{\hbar^2}{2m}(\Psi^*\Delta\Psi-\Psi\Delta\Psi^*)
  \right|_{\Delta = \vec{\nabla}\cdot\vec{\nabla}} =
  -\frac{\hbar^2}{2m} \vec{\nabla}(\Psi^*\vec{\nabla}\Psi-\Psi\vec{\nabla}\Psi^*) = \\ =
  -i\hbar \brc{-\frac{i\hbar}{2m}}\vec{\nabla}(\Psi^*\vec{\nabla}\Psi-\Psi\vec{\nabla}\Psi^*)
\end{gathered}
$$%
%
Поэтому из \eqref{eq:5_1_1} следует:

$$
  \boxed{
    \pd{\rho}{t} + \Div\vec{j} = 0
  }
$$%
%
где $\boxed{\vec{j}=-\frac{i\hbar}{2m}(\Psi^*\vec{\nabla}\Psi-\Psi\vec{\nabla}\Psi^*)}$.

Это соотношение называется {\em уравнением непрерывности} для плотности вероятности. В интегральной форме оно принимает вид:
$$
  \left.
    -\pd{}{t} \int_v \rho(\vr,t)\diff v = \int_v\Div\vec{j}\diff v
  \right|_{\text{т. Остр.-Гаусса}} = \oint_S \vec{j}\diff\vec{S}
$$%
%
$S$ -- поверхность, ограничивающая объём $v$. Интегральное уравнение показывает, что убыль вероятности $P =\int_v \rho(\vr,t)\diff v$ нахождения частицы в объёме $v$ равна потоку вероятности через границу объёма. Поэтому естественно истолковать вектор $\vec{j}(\vr,t)$ как {\em плотность потока (тока) вероятности}.

Плотность потока вероятности, вычисленная для волны де Бройля с единичной амплитудой
$$
\Psi_{\vp}(\vr, t) = e^{i(\vp\vr - Et)/\hbar}
$$%
%
определяется формулой $\vec{j}=p/m = \vec{v}$, т.е. совпадает с вектором классической скорости частицы.
\begin{excr}
Доказать вышеупомянутый факт.
\end{excr}%
%
Этот результат полезно использовать при расчёте {\em коэффициентов прохождения}
\begin{equation}
  \label{eq:5_1_2}
  \boxed{
    T(E) = \frac{
      \abs{\vec{j}_{\text{прош}}}
    }
    {
      \abs{\vec{j}_{\text{пад}}}
    }
  }
\end{equation}%
%
и {\em отражения}:

\begin{equation}
  \label{eq:5_1_3}
  \boxed{
    R(E) = \frac{
      \abs{\vec{j}_{\text{отр}}}
    }
    {
      \abs{\vec{j}_{\text{пад}}}
    }
  }
\end{equation}%
%
частицы в потенциальном поле $U(\vr)$. В формулах \eqref{eq:5_1_2} и \eqref{eq:5_1_3} $\vec{j}_{\text{прош}}$, $\vec{j}_{\text{отр}}$ и $\vec{j}_{\text{пад}}$ -- плотности потока вероятности прошедшей, отражённой и падающей волн соответственно. В случае одномерного движения частицы вдоль оси $x$:
$$
\Psi(\vr,t) \rightarrow \Psi(x,t)~;~~ U(\vr) \rightarrow U(x)
$$%
%
так что
\begin{equation}
\label{eq:5_1_4}
\boxed{
    j_x = -\frac{i\hbar}{2m} \brc{\Psi^*\pd{\Psi}{x} - \Psi\pd{\Psi^*}{x}}
    \frac{1}{m}\Re\left[\Psi^*(\op{p}_x\Psi)\right]
}
\end{equation}

\section{Оператор изменения во времени физической величины. Интегралы движения. Коммутаторы. Скобки Пуассона}

В общем случае оператор физической величины может явно зависеть от времени: $\op{F}=\op{F}(t)$. Пусть $\pd{\op{F}(t)}{t}$ -- производная по явной зависимости оператора от $t$. Среднее значение $F$ в общем случае также зависит от времени:

$$
  \avg{F} \equiv \bfkh{\Psi(t)}{\op{F}(t)}{\Psi(t)}
$$%
%
Пусть изменение во времени вектора состояния $\ket{\Psi(t)}$ описывается временным уравнением Шрёдингера\footnotemark:

\footnotetext{В соответствии с четвёртым постулатом квантовой механики в общем случае динамика квантовой системы, т.е. зависимость её волновой функции от времени, полностью определяется уравнением Шрёдингера}

\begin{equation}
  \label{eq:5_2_1}
  \boxed{
    i\hbar\pd{}{t}\ket{\Psi(t)} = \op{H}\ket{\Psi(t)}
  }
\end{equation}%
%
Пользуясь формулой \eqref{eq:5_2_1}, а также тем, что $\op{H}^+=\op{H}$, найдём производную среднего значения $\avg{F}$ по времени:

\begin{equation}
  \begin{split}
    \label{eq:5_2_2}
    \D{}{t} \avg{F} = \bfkh{\pd{\Psi}{t}}{\op{F}}{\Psi} 
        + \bfkh{\Psi}{\pd{\op{F}}{t}}{\Psi} + \bfkh{\Psi}{\op{F}}{\pd{\Psi}{t}} = \\ =
    \frac{i}{\hbar}\bfkh{\Psi}{\op{H}\op{F}}{\Psi} + \bfkh{\Psi}{\pd{\op{F}}{t}}{\Psi} -
        \frac{i}{\hbar}\bfkh{\Psi}{\op{F}\op{H}}{\Psi} = \\ =
    \bfkh{\Psi}{\pd{\op{F}}{t}+\frac{i}{\hbar}[\op{H},\op{F}]}{\Psi}
  \end{split}
\end{equation}

В выражении $\D{}{t} \avg{F}$ усреднение и оператор дифференцирования относятся к разным переменным, так что их можно переставить:
\begin{equation}
  \label{eq:5_2_3}
  \D{}{t} \avg{F} \equiv \avgh{\D{F}{t}}
\end{equation}%
%
Действительно:

$$
\left[
\begin{gathered}
  \avg{F} = \sum_n F_n P_n ~\rightarrow~ \D{}{t}\avg{F} =
    \sum_n \D{F_n}{t} P_n \equiv \avgh{\D{F}{t}}\\
  \avg{F} = \int F\diff P ~\rightarrow~ \D{}{t}\avg{F} =
    \int \D{F}{t}\diff P \equiv \avgh{\D{F}{t}}\\
\end{gathered}
\right.
$$%
%
Таким образом, производная $\D{F}{t}$ определена так, что её среднее значение равно производной по времени от среднего значения $\avg{F}$. Согласно общему определению \eqref{eq:2_2_1} оператора физической величины
\begin{equation}
  \label{eq:5_2_4}
  \avgh{\D{F}{t}} = \avgh{\D{\op{F}}{t}}_{\Psi} =
    \bfkh{\Psi}{\D{\op{F}}{t}}{\Psi}
\end{equation}%
%
Сравнивая \eqref{eq:5_2_3} и \eqref{eq:5_2_4} с \eqref{eq:5_2_2}, получим зависимость среднего значения физической величины от времени:

\begin{equation}
\label{eq:5_2_5}
  \boxed{
    \D{}{t} \avg{F} = \bfkh{\Psi}{\D{\op{F}}{t}}{\Psi}
  }
\end{equation}%
%
где
\begin{equation}
\label{eq:5_2_6}
\boxed{\D{\op{F}}{t} = \pd{\op{F}}{t} + \frac{i}{\hbar} [\op{H}, \op{F}]}
\end{equation}
-- оператор изменения физической величины во времени, или уравнение движения оператора $\op{F}$.

Из соотношений \eqref{eq:5_2_5} и \eqref{eq:5_2_6} следует, что если
\begin{itemize}
\item $\op{F}$ не зависит явно от времени $t$, т.е. $\partial\op{F}/\partial{t} = 0$;
\item $\op{F}$ коммутирует с гамильтонианом системы $\op{H}$, т.е. $[\op{H}, \op{F}]=0$.
\end{itemize}%
%
то среднее значение величины $F$ сохраняется во времени в произвольном состоянии $\ket{\psi}$, т.е. $\avg{F} = \const$. В таком случае говорят, что $F$ -- сохраняющаяся величина или {\em интеграл движения} для данной системы.

\begin{defn}
Величина, сохраняющая свое значение во времени, называется \underline{интегралом движения}.
\end{defn}%
%
Как видим, одним из условий для интеграла движения $F$ является условия совместной измеримости наблюдаемой $F$ с гамильтонианом $H$.

Приведём примеры интегралов движения:

\begin{itemize}
\item $\op{F} = \op{H}$ и $\op{H}$ не зависит от времени $t$ (гамильтониан замкнутой системы) -- полная энергия замкнутой системы сохраняется;
\item $\op{\vp} = - i \hbar \vec{\nabla}$.  Если $\displaystyle\op{H} = \frac{\op{\vp}^2}{2m}$ (свободное движение частицы массы $m$), то её импульс $\vp$ сохраняется.
\end{itemize}%
%
Подчеркнём, что одна и та же физическая величина $F$ в одних условиях может быть интегралом движения, а в других -- нет. Например, если частица с импульсом $\vp$ движется не свободно, а в потенциальном поле $U(\vr)$, то

\begin{equation}
\label{eq:5_2_7}
  [\op{H}, \op{\vp}] =
  \left.
    \brs{\frac{\op{\vp}^2}{2m}, \op{\vp}} + \brs{U(\vr), \vp}
  \right|_{\text{\cref{ex:1_1_7}}} =
  i \hbar \vec{\nabla} U(\vr) \ne 0
\end{equation}%
%
и её импульс $\vp$, в отличие от свободного движения, не является интегралом движения.

В классической механике мы имеем дело с обобщёнными координатами $\vec{q} = (q_1, ..., q_s)$, обобщёнными импульсами $\vp = (p_1, ..., p_s)$, а также с функциями обобщённых координат и импульсов $F = F(\vec{q}, \vp, t)$. Здесь $s$ -- количество независимых степеней свободы в механической системе. Полная производная по времени от величины $F$ определяется соотношением

$$
\D{F}{t} = \pd{F}{t} + \sum_{i=1}^{s}\brc{\pd{F}{q_i} \dot{q_i} + \pd{F}{p_i} \dot{p_i} }
$$%
%
Подставляя, согласно уравнениям Гамильтона (см.~\llref{40}{1}, уравнения Гамильтона)

$$
\dot{q_i} = \pd{H}{p_i} ~~~ \dot{p_i} = -\pd{H}{q_i}
$$%
%
получим

\begin{equation}
  \label{eq:5_2_8}
  \D{F}{t} = \pd{F}{t} + \underbrace{
    \sum_{i=1}^{s} \brc{ \pd{H}{p_i} \pd{F}{q_i} - \pd{H}{q_i} \pd{F}{p_i} }
  }_{\brcr{H, F}} = \pd{F}{t} + \brcr{H, F}
\end{equation}%
%
где $\brcr{H, F}$ -- скобка Пуассона\footnotemark для величин $H$ и $F$ (см.~\llref{42}{1}).
%people
\footnotetext{Симеон Дени Пуассон (Sim\'{e}on Denis Poisson, 1781-1840)}

Если неопределённость $\Delta F$ физической величины $F$ мала по сравнению с $\avg{F}$, то среднее значение $\avg{F}$ должно меняться по тому же закону, что и классическое значение $F$ ({\em принцип соответствия} классических уравнений движения и усреднённых квантово-механических уравнений, т.е. уравнения для описания эволюции во времени квантово-механических средних или наблюдаемых на опыте величин). Сравнивая \eqref{eq:5_2_8} с \eqref{eq:5_2_5} и \eqref{eq:5_2_6}, получаем следующие соответствия:

$$
\begin{aligned}
F ~~&\leftrightarrow~~ \avg{F} \\
\pd{F}{t} ~~&\leftrightarrow~~ \bfkh{\Psi}{\pd{\op{F}}{t}}{\Psi} \\
\brcr{H, F} ~~&\leftrightarrow~~ \frac{i}{\hbar} \bfk{\Psi}{ [\op{H}, \op{F}] }{\Psi}
\end{aligned}
$$%
%
По этой причине коммутатор $[\op{H}, \op{F}]$ иногда называют {\em квантовой скобкой Пуассона}. При переходе к классическому пределу ($\hbar \to 0$) оператор $i[\op{H}, \op{F}]$ стремится к $\hbar \brcr{H, F}$. Для эрмитовости квантовой скобки Пуассона при ней вводится $i$. Указанное соответствие между коммутатором и скобкой Пуассона может быть использовано для определения явного вида операторов физических величин $F$.

\section{Производная по времени операторов координаты и импульсов частицы в потенциальном поле. Теоремы Эренфеста}

Координаты и импульсы являются независимыми переменными, не зависящими явно от времени. Поэтому, согласно \eqref{eq:5_2_6}, операторы производных этих величин по времени выражаются просто через квантовые скобки Пуассона. Итак, имеем:

$$
  \op{\vec{v}} \equiv \D{\op{\vr}}{t} = \frac{i}{\hbar} \brs{\op{H}, \op{\vr}} =
    \frac{i}{\hbar} \brs{\op{T}, \op{\vr}} + \frac{i}{\hbar} \underbrace{
      \cancel{\brs{U(\op{\vr}), \op{\vr}}}
    }_{= 0}
$$%
%
где $\op{T}$ -- оператор кинетической энергии. Второй коммутатор равен нулю, поскольку оператор координаты $\op{\vr}$ коммутирует сам с собой, с любой степенью $\op{\vr}^n$, а значит, с произвольной функцией оператора координаты:

$$
U(\op{\vr}) \equiv \sum_{n=0}^{\infty} \frac{U^{n}(0)}{n!}\op{\vr}^n ~~~ \to ~~~ \brs{U(\op{\vr}), \op{\vr}} = 0
$$%
%
Осталось вычислить коммутатор с оператором кинетической энергии:
$$
  \brs{\op{T}, \op{\vr}} = \brs{\frac{\op{\vp}^2}{2m}, \op{\vr}} =
  -i \hbar \frac{\op{\vp}}{m}
$$%
%
Окончательно для оператора скорости частица получаем
\begin{equation}
\label{eq:5_3_1}
  \op{\vec{v}} = \frac{\op{\vp}}{m}
\end{equation}%
%
Аналогично, определим оператор силы:

\begin{equation}
\label{eq:5_3_2}
  \op{\vec{F}} \equiv \left. \D{\op{\vp}}{t} \right|_{
    \text{\eqref{eq:5_2_6}}
  } = \left. \frac{i}{\hbar} \brs{\op{H}, \op{\vp}} \right|_{
    \text{\eqref{eq:5_2_7}}
  } = - \vec{\nabla} U(\op{\vr})
\end{equation}%
%
Согласно уравнению \eqref{eq:5_2_5}, операторы $\D{\vr}{t}$ и $\D{\vp}{t}$ определяют скорости изменения средних значений координаты $\avg{\op{\vr}}$ и импульса $\avg{\op{\vp}}$ частицы, соответственно:

\begin{equation}
\label{eq:5_3_3}
\boxed {
	\left. \D{}{t}\avg{\op{\vr}} \right|_{\text{\eqref{eq:5_3_1}}} = \frac{\avg{\op{\vp}}}{m}
}
\end{equation}

\begin{equation}
\label{eq:5_3_4}
\boxed {
	\left. \D{}{t}\avg{\op{\vp}} \right|_{\text{\eqref{eq:5_3_2}}} = - \avg{\vec{\nabla} U(\op{\vr})}
}
\end{equation}

Из \eqref{eq:5_3_3}, \eqref{eq:5_3_4} получается квантово-механический аналог уравнения Ньютона:
\begin{equation}
\label{eq:5_3_5}
m \D{^2}{t^2} \avg{\op{\vr}} = \avg{\op{\vec{F}}}
\end{equation}

Результаты с \eqref{eq:5_3_3} по \eqref{eq:5_3_5} -- {\em теоремы Эренфеста}\footnotemark (1927 г.) -- показывают, что средние значения квантовых величин (операторов координаты и импульса) подчиняются уравнениям классической механики. Эренфест доказал предельное соответствие (принцип соответствия) квантовой и классической механик, а именно: уравнения движения классической механики есть предельный случай более общих уравнений квантовой механики.
%people
\footnotetext{Пауль Эренфест (Paul Ehrenfest, 1880-1933)}

Однако найденные Эренфестом уравнения не тождественны ньютоновским. Действительно, стоящая в правой части \eqref{eq:5_3_5} средняя сила, вообще говоря, не совпадает с классической силой, действующей в точке $\avg{\op{\vr}}$:

\begin{equation}
\label{eq:5_3_6}
\avg{\op{\vec{F}}(\vr)} \neq \vec{F}(\avg{\op{\vr}})
\end{equation}%
%
Для левой и правой частей \eqref{eq:5_3_6} имеем:

\begin{equation}
\begin{aligned}
\label{eq:5_3_7}
\avg{\op{\vec{F}}(\vr)} = &
	\left\langle \op{F}(\avg{\vr}) + \brc{\op{\vr} - \avg{\op{\vr}}} \nabla \vec{F}(\avg{\op{\vr}})~+ \right. \\ & \left.+~
	\frac{1}{2} (\op{r}_\alpha - \avg{\op{r}_\alpha})(\op{r}_\beta - \avg{\op{r}_\beta}) \cdot \pd{^2 \vec{F}(\avg{\op{\vr}})}{r_\alpha \partial r_\beta} + ... \right\rangle
\end{aligned}  
\end{equation}%
%
При усреднении член с $\nabla \vec{F}$ в \eqref{eq:5_3_7} исчезает, так что

\begin{equation}
\label{eq:5_3_8}
\avg{\op{\vec{F}}(\vr)} \approx \vec{F}(\avg{\op{\vr}}) + \frac{1}{2} \avgh{(\op{r}_\alpha - \avg{\op{r}_\alpha})(\op{r}_\beta - \avg{\op{r}_\beta})} \times \pd{^2 \vec{F}(\avg{\op{\vr}})}{r_\alpha \partial r_\beta}
\end{equation}%
%
Из \eqref{eq:5_3_8} следует, что свободное движение $(\vec{F} = 0)$, движение в однородном поле $(\vec{F} = \const)$ или в поле упругой силы (гармонический осциллятор с $\vec{F} = - m\omega^2 \vr$) всегда классичны в смысле выполнения точного равенства:

\begin{equation}
\label{eq:5_3_9}
  \left. \avgh{\op{\vec{F}}(\vr)} \right|_{
    \eqref{eq:5_3_5}
  } = m \D{^2}{t^2} \avg{\op{\vr}} = \vec{F}(\avg{\op{\vr}})
\end{equation}%
%
В остальных случаях равенство \eqref{eq:5_3_9} выполняется приближённо только если волновая функция частицы кализована в достаточно малой области пространства $\Delta r$ (размер области, где плотность вероятности $\abs{\Psi}^2$ существенно отлична от нуля) по сравнению с размером $L$ области существенного изменения силы.

\begin{equation}
\label{eq:5_3_10}
\Delta r \ll L
\end{equation}%
%
При {\em условии классичности движения} \eqref{eq:5_3_10} вторым слагаемым в \eqref{eq:5_3_8} можно пренебречь, и тогда движение области локализации частицы определится вторым законом Ньютона.

Теоремы Эренфеста позволяют понять, почему движение электрона в электронно-лучевой трубке описывается классическими уравнениями, тогда как движение электрона в атоме -- квантовыми.
