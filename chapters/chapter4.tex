\chapter{Совместная измеримость физических величин}

\section{Условия одновременной измеримости физических величин. Коммутаторы.}

До сих пор изучалось вероятностное распределение результатов измерения одной наблюдаемой $\op{F}$. Пусть теперь мы имеет пару наблюдаемых $\op{F}$ и $\op{G}$ с дискретными спектрами:

$$
\begin{array}{lcl}
\op{F}, \op{G}:  & F \rightarrow \op{F} \rightarrow \{f_n\}, \; n=0,1,2,\dots\\
                 & G \rightarrow \op{G} \rightarrow \{g_m\}, \; m=0,1,2,\dots
\end{array}
$$%
%
Для простоты будем рассматривать только дискретные спектры. Пусть имеется состояние $\ket{\phi}$ со свойствами:

\begin{equation}
\label{eq:4_1_1}
  \begin{cases}
    \op{F} |\phi \rangle = f_n |\phi \rangle\\
    \op{G} |\phi \rangle = g_m |\phi \rangle
  \end{cases} 
\end{equation}%
%
т.е. $\ket{\phi}$ -- {\em общий собственный вектор} наблюдаемых $\op{F}$ и $\op{G}$. Тогда, согласно общим положениям, в состоянии $\ket{\phi}$ достоверно известно, что $F = f_n$ и $G=g_m$, значит, можно упростить обозначения: $\ket{\phi} \equiv \ket{f_n g_m} \equiv \ket{nm}$

Предположим, что у системы уравнение имеется такое множество решений, что система векторов $\brcr{\ket{nm}}$ полна, и по ней можно разложить любой вектор состояния:

\begin{equation}
  \label{eq:4_1_2}
  \forall \ket{\psi} \in \mathcal{H} ~\rightarrow ~ \ket{\psi} =
      \underbrace{\sum_{n,m}\ket{nm}\bra{nm}}_{=\mathds{1}} \psi \rangle 
\end{equation}%
%
При этом коэффициенты разложения трактуются как амплитуды совместных вероятностей
$$
  P_{\ket{\psi}}(F = f_n, G = g_m) = \abs{\bk{nm}{\psi}}^2
$$%
%
Физически предположение о полноте означает, что у экспериментатора имеется универсальная возможность одновременно (совместно) измерить физические величины $F$ и $G$, т.е. можно создать универсальный прибор для измерения пары наблюдаемых. При этом измерения одной из величин никак не скажется на измерении другой.

\begin{defn}
Физические величины $F$ и $G$ одновременно (совместно) измеримы, если их операторы $\op{F}$ и $\op{G}$ обладают общей полной системой собственных векторов (собственных функций).
\end{defn}

Одновременная измеримость двух физических величин налагает весьма жёсткие условия на их операторы.

\begin{thm}\label{theorema_iv_chapter}
Для того, чтобы физические велчины $F$ и $G$ были совместно измеримы, необходимо и достаточно, чтобы операторы $\op{F}$ и $\op{G}$ коммутировали, то есть: 
$$[\op{F}, \op{G}] \equiv \op{F}\op{G} - \op{G}\op{F} = 0$$
\end{thm}

\begin{proof}

{\bf Необходимость.} Пусть $F$ и $G$ совместно измеримы. Тогда, по определению для их операторов существует полная общая система собственных векторов $\brcr{\ket{nm}} \in \mathcal{H}$ и выполняется следующее:

$$
\begin{gathered}
\forall \ket{\psi} \in \mathcal{H} ~\rightarrow ~
  (\op{F}\op{G} - \op{G}\op{F}) \ket{\psi} =
    \sum_{n,m} (\op{F}\op{G} - \op{G}\op{F}) \ket{nm}\bk{nm}{\psi} = \\ =
    \sum_{n,m} (f_n g_m - g_m f_n) \ket{nm}\bk{nm}{\psi} = 0
\end{gathered}
$$%
%
Что и требовалось доказать. Доказано утверждение, что если у $\op{F}$ и $\op{G}$ есть общая полная система собственных векторов, то результат действия коммутатора $[\op{F}, \op{G}]$ на $\forall \ket{\psi} \in \mathcal{H}$ равен нулю, т.е. $[\op{F}, \op{G}] = 0$

{\bf Достаточность.} Если операторы коммутируют, то при определённых оговорках на их область определения у них есть общая полная система собственных векторов.

\begin{enumerate}
\renewcommand\labelenumi{(\alph{enumi})}

\item Предположим, что $[\op{F}, \op{G}] = 0$, и, кроме того, что спектр оператора $\op{F}$ -- невырожденный, т.е. каждому собственному значению $f_n$ отвечает только один собственный вектор $\ket{n}$:

\begin{equation}
  \label{eq:4_1_3}
  \op{F} \ket{n} = f_n \ket{n}
\end{equation}%
%
Поэтому
$$
  \op{G}(\op{F} \ket{n}) =
    \left. \op{F}(\op{G}\ket{n}) \right|_{(4.1.3)} =
    f_n (\op{G}\ket{n})
$$%
%
Таким образом, вектор $\op{G} \ket{n}$ является собственным для оператора $\op{F}$ и отвечает тому же собственному значению $f_n$. Поскольку спектр оператора $\op{F}$ предполагается невырожденным, то векторы $\ket{n}$ и $\op{G} \ket{n}$ должны быть коллинеарны, т.е. могут отличаться лишь числовым множителем, отличным от нуля:
$$
\op{G} \ket{n} = g_m \ket{n}
$$%
%
Следовательно, состояния $\ket{n}$ оказываются собственными и для оператора $\op{G}$. В этих состояниях наблюдаемая $\op{G}$ принимает определённые значения $g_m$, т.е. система собственных векторов $\ket{n} = \ket{f_n g_m} \equiv \ket{nm}$, полная по предположению, является общей для операторов $\op{F}$ и $\op{G}$.

\item Случай вырожденного спектра оператора $\op{F}$ будет доказан в \sref{2}{6}.
\end{enumerate}
\end{proof}

Условия теоремы не запрещают двух некоммутирующим наблюдаемым ($[\op{F},\op{G}] \ne 0$) иметь общий собственный вектор. Но даже если таких векторов несколько, но {\em мало}, то из них нельзя построить универсальный прибор, чтобы измерить совместно физические величины $F$ и $G$. Можно построить лишь специальный прибор\footnote{Что бы это ни значило}.

В заключение ещё раз обсудим содержание доказанной теоремы. Пусть спектр оператора $\op{F}$ вырожден, т.е. одному собственному значению отвечает сразу несколько собственных функций (собственных векторов.)

\begin{defn}
Число $f$, входящее в собственные значения оператора физической величины $\op{F}$, называют \underline{квантовым числом}, характеризующим состояние системы $\ket{\psi_f} \equiv \ket{f}$.
\end{defn}%
%
Иными словами, в случае вырожденного спектра квантовое число $f$ не определяет однозначно квантовое состояние системы. В этом случае всегда существуют взаимно коммутирующие операторы $\op{G}_1, \op{G}_2...$ (в частном случае, один оператор $\op{G}$), коммутирующие с $\op{F}$. Любая собственная функция $\psi_{f g_1 g_2 ...}(\vec{q})$ этих операторов или собственный вектор $\ket{f g_1 g_2 ...}$ характеризуется определённым набором квантовых чисел $f, g_1, g_2, ...$, которые однозначно фиксируют квантовое состояния.

\begin{defn}
Набор взаимно коммутирующих операторов, собственные значения которых однозначно определяют квантовое состояние системы, называют \underline{полным набором совместных наблюдаемых}.
\end{defn}

\section{Соотношение неопределённостей}

Пусть $\op{F}$ и $\op{G}$ -- операторы физических величин $F$ и $G$, т.е. $\op{F}^\dag=\op{F}$ и $\op{G}^\dag=\op{G}$, и их коммутатор имеет вид $[\op{F},\op{G}]=i\op{K}$.

\begin{thm}
В произвольном квантовом состоянии выполняется соотношение неопределённостей:
$$
  \boxed{
    \avgh{\brc{\op{F}-\avg{\op{F}}}^2} \avgh{\brc{\op{G}-\avg{\op{G}}}^2} \geqslant
      \frac{\avg{\op{K}}^2}{4}}
$$
\end{thm}

\begin{proof}
\begin{enumerate}
\item Покажем что если коммутатор представим в виде $[\op{F},\op{G}]=i\op{K}$, то оператор $\op{K}$ -- эрмитов ($\op{K}^\dag=\op{K}$). Действительно,
\begin{equation}
  \label{eq:4_2_1}
  [\op{F},\op{G}]^\dag=-i\op{K}^\dag
\end{equation}%
%
С другой стороны:

\begin{equation}
\begin{gathered}
  \label{eq:4_2_2}
  [\op{F},\op{G}]^\dag= (\op{F}\op{G} - \op{G}\op{F})^\dag =
    \op{G}^\dag \op{F}^\dag - \op{F}^\dag \op{G}^\dag = \\ =
    -[\op{F}^\dag,\op{G}^\dag]= -[\op{F},\op{G}]= -i\op{K}
\end{gathered}
\end{equation}%
%
Из сравнения правых частей \eqref{eq:4_2_1} и \eqref{eq:4_2_2} следует что $\op{K}^\dag=\op{K}$

\item Введём в состоянии $\ket{\psi}$ оператор отклонения от среднего $\avg{F}=\bfk{\psi}{\op{F}}{\psi}$, т.е.
%
$$
  \Delta\op{F}=\op{F}-\avg{\op{F}}\cdot\mathds{1}
$$%
%
Аналогично $\Delta\op{G} = \op{G} - \avg{\op{G}} \cdot \mathds{1}$. Очевидно, что $[\Delta\op{F}, \Delta\op{G}] = i\op{K}$.

\item Введём в рассмотрение вектор состояния
$$
  \ket{\phi}=(\Delta\op{F}-i\gamma\Delta\op{G})\ket{\psi}
$$%
%
где $\gamma$ -- вещественный параметр. Сопрягаем:
$$
  \bra{\phi} = \bra{\psi} (\Delta\op{F} - i\gamma\Delta\op{G})^\dag =
    \bra{\psi} (\Delta\op{F} + i\gamma\Delta\op{G})
$$%
%
При этом мы учли, что $\op{F}^\dag = \op{F}$, $\op{G}^\dag = \op{G}$, а также вещественность средних значений эрмитовых операторов: $\avg{\op{F}}^* = \avg{\op{F}}$, $\avg{\op{G}}^* = \avg{\op{G}}$.

\item Рассмотрим
$$
\begin{gathered}
  \bk{\phi}{\phi} = \norm{\ket{\phi}} =
    \bra{\psi}(\Delta\op{F} + i\gamma\Delta\op{G})
        (\Delta\op{F} - i\gamma\Delta\op{G})\ket{\psi} = \\ =
    \bfk{\psi}{\Delta\op{F}^2}{\psi} + \gamma^2\bfk{\psi}{\Delta\op{G}^2}{\psi} +
      \gamma\bfk{\psi}{\op{K}}{\psi} \geqslant 0
\end{gathered}
$$%
%
Неравенство выполняется, т.к. норма любого вектора состояний положительно определена. Мы получим неотрицательный квадратный трёхчлен по $\gamma$. Для выполнения последнего неравенства необходимо, чтобы дискриминант квадратного трёхчлена был отрицательным или равным нулю, т.е.
$$
  \avg{\op{K}}^2 - 4 \avg{\Delta\op{F}^2} \avg{\Delta\op{G}^2} \leqslant 0
$$%
%
Отсюда следует общее соотношение неопределённостей для дисперсий двух неизмеримых совместно величин:
$$
  \boxed{
    \avg{\Delta\op{F}^2} \avg{\Delta\op{G}^2} \geqslant \frac{\avg{\op{K}}^2}{4}
  }
$$

Напомним, что дисперсия характеризует меры отклонения результатов измерения от среднего значения (или неопределённость в измерении величин):
$$
  \avgh{\Delta\op{F}^2} = \avgh{(\op{F}-\avg{\op{F}})^2} =
    \avgh{\op{F}^2}-\avgh{\op{F}}^2
$$
\end{enumerate}
\end{proof}

\begin{exmpl*} Пусть $\op{F}=\op{x}=x$, а $\op{G}=\op{p}_x=-i\hbar\pd{}{x}$. Вычислим коммутатор:
$$
  [\op{x}, \op{p}_x] \psi(x) =
    x (-i\hbar\pd{}{x})\psi(x) - (-i\hbar\pd{}{x}) x \psi(x) =
    i\hbar\psi(x)
$$%
%
Значит $\op{K}=\hbar$, и соотношение неопределённостей для координаты и импульса имеет вид:
$$
  \boxed{\avgh{(\Delta\op{x})^2}\avgh{(\Delta\op{p}_x)^2} \geqslant \frac{\hbar^2}{4}}
$$
\end{exmpl*}
