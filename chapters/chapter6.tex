\chapter{Теория представлений}

\section{Матричное представление}

Пусть спектр оператора $\op{G}$ дискретный.

$$\op{G} \ket{n} = g_n \ket{n},~~~\text{где}~\ket{n} \equiv \ket{\psi_n},~n \in \mathbb{N}$$

Здесь $n$ - индекс состояния.

Из \eqref{eq:3_3_11} (условие полноты):
$$
\sum_n \ket{n}\bra{n} = \mathds{1}
$$

Разложим по базису оператора $\op{G}$:
$$
\ket{\psi}=\sum_n \ket{n}\bk{n}{\psi}
$$

В $g$-представлении:
$$
\bk{n}{\psi} = \begin{pmatrix}
\bk{1}{\psi}\\
\bk{2}{\psi}\\
...
\end{pmatrix}
$$
\begin{equation}
\begin{split}
\label{eq:6_1_1}
\bk{n}{\op{F}\psi} = \bkh{n}{\op{F}\brc{\sum_{n'} \ket{n'}\bra{n'}} \psi} = \sum_{n'} \bfk{n}{\op{F}}{n'}\bk{n'}{\psi} = \\ = \begin{pmatrix}
F_{11} & F_{12} & \cdots\\
F_{21} & F_{22} & \cdots\\
\cdots & \cdots & \cdots
\end{pmatrix} \begin{pmatrix}
\bk{1}{\psi}\\
\bk{2}{\psi}\\
...
\end{pmatrix}
\end{split}
\end{equation}

$F_{nn'} = \bfk{n}{\op{F}}{n'}$ --- матричное представление $\op{F}$ в базисе состояний $\{\ket{n}\}$ собственных векторов $\op{G}$ (оператор $\op{F}$ в $g$-представлении). Если $\op{G}=\op{H}$, то это $E$-представление.

Матричное представление эрмитово-сопряжённого оператора:
$$
\left .\bfk{n'}{\op{F}^\dag}{n}\right|_{(3.2.6)} = \bfk{n}{\op{F}}{n'}^*
$$
т.е. $\brc{F^\dag}_{n'n}=F^*_{nn'}$

Эрмитово сопряжение := транспонирование + комплексное сопряжение.

Если $\op{F}^\dag=\op{F}$, то $F^*_{nn'}$ --- эрмитова матрица

В случае $n'=n$: $F_{nn}=F_{nn}^*$ (т.е. диагональные матричные элементы эрмитовой матрицы вещественны)
$$
\bfk{n}{\op{A}\op{B}}{n''} = \sum_{n'} \bfk{n}{\op{A}}{n'} \bfk{n'}{\op{B}}{n''} = \sum_{n'} A_{nn'} B_{n'n''} \text{(произведение матриц)}
$$

Задача на собственные значения и векторы:
$$
\op{F}\ket{f}=f\ket{f} \xrightarrow{(6.1.1)} \sum_{n'} F_{nn'} \bk{n'}{f} = f\bk{n}{f}=f\delta_{nn'}\bk{n'}{f}
$$
$$
\sum_{n'}\brc{F_{nn'}-f\delta_{nn'}}\bk{n'}{f}=0
$$

Корни $f$ характеристического уравнения являются собственными значениями оператора:
$$
\det{\norm{F_{nn'}-f\delta_{nn'}}}=0
$$

В $f$-представлении оператора $\op{F}$:
$$
\bfk{f'}{\op{F}}{f}=f\bk{f'}{f}=f\delta_{f'f}
$$
т.е. $F$ --- диагональная матрица

\section{Унитарное преобразование векторов-состояний и операторов}
$$
\begin{gathered}
\op{L}\ket{\lambda}=\lambda\ket{\lambda},~ \ket{\lambda}=\ket{\chi_\lambda} \\
\op{M}\ket{\mu}=\mu\ket{\mu},~ \ket{\mu}=\ket{\phi_\mu}
\end{gathered}
$$
$$
\ket{\psi}=\sum_\lambda \ket{\lambda}\bk{\lambda}{\psi}=\sum_\mu \ket{\mu}\bk{\mu}{\psi}
$$
Обозначим:
$$
\begin{gathered}
\bk{\lambda}{\psi}=\bk{\chi_\lambda}{\psi}=\psi_\lambda \\
\bk{\mu}{\psi}=\bk{\phi_\mu}{\psi}=\psi'_\mu
\end{gathered}
$$
(волновая функция в $\lambda$ и $\mu$ представлениях)

\begin{equation}
\label{eq:6_2_1}
\psi'_\mu=\bk{\mu}{\psi}=\sum_\lambda \underbrace{\bk{\mu}{\lambda}}_{U_{\mu\lambda}}\bk{\lambda}{\psi} = \sum_\lambda U_{\mu\lambda}\psi_\lambda
\end{equation}
$$
\begin{gathered}
\bra{\mu}=\sum_\lambda \underbrace{\bk{\mu}{\lambda}}_{U_{\mu\lambda}} \bra{\lambda} = \sum_\lambda U_{\mu\lambda}\bra{\lambda} \\
\ket{\mu}=\sum_\lambda \ket{\lambda} \bk{\lambda}{\mu} = \sum_\lambda U^*_{\mu\lambda}\ket{\lambda} 
\end{gathered}
$$

Условия ортонормировки:
$$
\begin{gathered}
\bk{\mu}{\mu'}=\delta_{\mu\mu'}\\
\bk{\lambda}{\lambda'}=\delta_{\lambda\lambda'}
\end{gathered}
$$

$$
\delta_{\mu\mu'} = 
\bk{\mu}{\mu'} =
\sum_\lambda \sum_{\lambda'} \underbrace{\bk{\mu}{\lambda}}_{U_{\mu\lambda}}
\underbrace{\bk{\lambda}{\lambda'}}_{\delta_{\lambda\lambda'}}
\underbrace{\bk{\lambda'}{\mu'}}_{\brc{U^\dag}_{\lambda'\mu'}} =
\sum_\lambda U_{\mu\lambda}\brc{U^\dag}_{\lambda\mu'}
$$

\begin{defn}
$\op{U}$ называется \underbar{унитарным}, если $\op{U}^\dag = \op{U}^{-1}$, т.е. $\op{U}^\dag\op{U}=\op{U}\op{U}^\dag=\mathds{1}$
\end{defn}

Если $\op{U}^\dag\op{U}=\mathds{1}$, то:
$$
\sum_{n'}U_{nn'} \brc{U^\dag}_{n'n''}=\delta_{nn''}~ \Rightarrow~ U^\dag=U^{-1} 
$$
(т.е. $U$ --- унитарная матрица)
\begin{equation}
\label{eq:6_2_2}
\ket{\psi'}=\op{U}\ket{\psi}
\end{equation}
Преобразование вектора-состояния из одного представления в другое является унитарным. 

Пусть $\op{F}$ -- оператор физической величины в $\lambda$-представлении, а $\op{F'}$ --\\
в $\mu$-представлении.

$$
\bk{\mu'}{\lambda'}^*=U^*_{\mu'\lambda'}=\brc{U^\dag}_{\lambda'\mu'}
$$
\begin{equation}
\label{eq:6_2_3}
F_{\mu\mu'} = 
\bfk{\mu}{\op{F}}{\mu'} =
\sum_\lambda \sum_{\lambda'} 
\underbrace{\bk{\mu}{\lambda}}_{U_{\mu\lambda}}
\underbrace{\bfk{\lambda}{\op{F}}{\lambda'}}_{F_{\lambda\lambda'}}
\bk{\lambda'}{\mu'} =
\sum_\lambda \sum_{\lambda'} U_{\mu\lambda} F_{\lambda\lambda'} \brc{U^\dag}_{\lambda'\mu'}
\end{equation}

\begin{equation}
\label{eq:6_2_4}
\boxed{F'=U F U^\dag}
\end{equation}

\begin{equation}
\label{eq:6_2_5}
\boxed{F=U^\dag F' U}
\end{equation}

\begin{equation}
\label{eq:6_2_4_add}
\boxed{\op{F'}=\op{U} \op{F} \op{U}^\dag}
\tag{\ref{eq:6_2_4}$'$}
\end{equation}

\begin{equation}
\label{eq:6_2_5_add}
\boxed{\op{F}=\op{U}^\dag \op{F'} \op{U}}
\tag{\ref{eq:6_2_5}$'$}
\end{equation}

В классе унитарных преобразований векторов и операторов справедливы следующие утверждения:
\begin{enumerate}
\item Скалярное произведение любых двух векторов инвариантно к унитарным преобразованиям:
$$
\bk{\psi'_1}{\psi'_2}=\bk{\psi_1}{\psi_2}
$$
\item Унитарные преобразования не меняют собственные значения эрмитовых операторов: если $\op{F}\ket{f}=f\ket{f}$, то $\boxed{\op{F'}\ket{f'}=f\ket{f'}}$

\item Унитарные преобразования не нарушают эрмитовости операторов: если $\op{F}^\dag=\op{F}$, то $(\op{F'})^\dag=\op{F'}$

\item Унитарные преобразования сохраняют коммутационные соотношения: \\
если $[\op{F}, \op{G}]=\op{K}$, то $[\op{F'}, \op{G'}]=\op{K'}$

\item $\boxed{\bfk{\psi_1}{\op{F}}{\psi_2}=\bfk{\psi'_1}{\op{F'}}{\psi'_2}}$
\end{enumerate}

\begin{excr}
Доказать утверждения 1-5.
\end{excr}

Физическое содержание теории при унитарных преобразованиях не меняется.

% TODO: сделать полноценные библиографические ссылки

Аналогия в классической механике --- канонические преобразования (см. Л.-Л. т.I Теоретическая механика, \textsection 45).

Продолжение доказательства \hyperref[theorema_iv_chapter]{теоремы} из лекции 4:
$[\op{F},\op{G}]=0$, $\op{F}$ --- вырожденный.

Соответствующие собственные значения $f_n \rightarrow \{\ket{\psi_n^{(i)}}\} \equiv \ket{n^{(i)}},~i = 1..k$
$$
\begin{gathered}
\left\{\ket{n^{(i)}}\right\} \xrightarrow{U} \left\{\ket{n'^{(i)}}\right\} \\
\ket{n'^{(i)}}=\sum_j U_{ij} \ket{n^{(j)}} \\
G'=U G U^\dag
\end{gathered}
$$

Подходящим унитарным преобразованием любая эрмитовая матрица приводится к диагональному виду:
$$
G'_{ij}=g_i \delta_{ij}~~~\text{или}~~~\op{G}\ket{n'^{(i)}}=g_i\ket{n'^{(i)}}
$$

$\ket{n'^{(1)}}, \ket{n'^{(2)}}... \ket{n'^{(k)}}$ --- собственные векторы

Следовательно, величины совместно измеримы. (доказана достаточность теоремы)

\section{Координатное и импульсное представления}

Пусть $\vr \rightarrow \ket{\vr}$ (а значит $\vr \rightarrow \op{\vr}$) \\
тогда $\vp \rightarrow \ket{\vp}$

Задача на собственные значения:
\begin{equation}
\label{eq:6_3_1}
\op{\vr}\ket{\vsr}=\vsr\ket{\vsr}
\end{equation}

\begin{equation}
\label{eq:6_3_2}
\op{\vp}\ket{\vsp}=\vsp\ket{\vsp}
\end{equation}

Подействуем справа $\bra{\vec{p''}}$:
$$
\bfkh{\vec{p''}}{\op{\vp}}{\vsp}=\vsp\bkh{\vec{p''}}{\vsp}
$$
Из \eqref{eq:3_4_3}:
$$
\bkh{\vec{p''}}{\vsp}=\delta(\vec{p''}-\vsp)
$$

Таким образом, <<матрица>> оператора импульса в импульсном ($p$-) представлении:
\begin{equation}
\label{eq:6_3_3}
\boxed{\bfkh{\vec{p''}}{\op{\vp}}{\vsp}=\vsp \delta(\vec{p''}-\vsp)}
\end{equation}

Аналогично:
\begin{equation}
\label{eq:6_3_4}
\boxed{\bfkh{\vec{r''}}{\op{\vr}}{\vsr}=\vsr \delta(\vec{r''}-\vsr)}
\end{equation}

Из гл.3 \textsection 3:
\begin{equation}
\label{eq:6_3_5}
\op{P}_{\vr}=\ket{\vr}\bra{\vr}
\end{equation}

\begin{equation}
\label{eq:6_3_6}
\op{P}_{\vr} \ket{\psi} = \ket{\vr}\bkh{\vr}{\psi} = \bkh{\vr}{\psi} \ket{\vr}
\end{equation}

\begin{equation}
\label{eq:6_3_7}
\bkh{\vr}{\psi} \equiv \psi(\vr)
\end{equation}

Из гл.3 \textsection 4:
$$
\forall \ket{\psi} \in \mathcal{H} ~~ \left\{ \ket{\vr} \right\}
$$
\begin{equation}
\label{eq:6_3_8}
\ket{\psi}=\int \ket{\vr}\bkh{\vr}{\psi} \, d\vr
\end{equation}

\begin{equation}
\label{eq:6_3_8_add}
\ket{\Phi}=\int \ket{\vp}\bkh{\vp}{\Phi} \, d\vp
\tag{\ref{eq:6_3_8}$'$}
\end{equation}

\begin{equation}
\label{eq:6_3_7_add}
\boxed{\bkh{\vp}{\Phi} \equiv \Phi\brc{\vp}}
\tag{\ref{eq:6_3_7}$'$}
\end{equation}

$\abs{\Phi\brc{\vp}}^2 d\vp$ --- вероятность обнаружить частицу с импульсом в интервале $[\vp, \vp+d\vp]$

Пусть $\ket{\psi}\equiv\ket{\vp}$, тогда из равенства \eqref{eq:6_3_6}:
\begin{equation}
\label{eq:6_3_6_add}
\op{P}_{\vr}\ket{\vp}=\bk{\vr}{\vp}\ket{\vr}
\tag{\ref{eq:6_3_6}$'$}
\end{equation}

Волна де Бройля:
\begin{equation}
\label{eq:6_3_9}
\boxed{
	\left. \Psi_{\vp}(\vr,t) \right|_{\text{\eqref{eq:2_1_2}}} = 
	\left. \frac{1}{(2\pi \hbar)^{(3/2)}} e^{\frac{i}{h} (\vp\vr-Et)}  \right|_{\text{\eqref{eq:6_3_6_add}}} = 
	\bk{\vr}{\vp} e^\frac{-iEt}{\hbar}
}
\end{equation}

\begin{equation}
\label{eq:6_3_10}
\psi(\vr) \equiv \bk{\vr}{\psi} = 
\int \underbrace{  \bk{\vr}{\vp}  }_ {\psi_{\vp}(\vr)}  \underbrace{ \bk{\vp}{\psi} }_{\psi(\vp)} \, d\vp =
\int \psi_{\vp}(\vr) \psi(\vp) \, d\vp
\end{equation}

Формула \eqref{eq:6_3_10} --- аналог \eqref{eq:2_1_4}. Также можно записать:
\begin{equation}
\label{eq:6_3_11}
\psi(\vp) \equiv \bk{\vp}{\psi} = 
\int \underbrace{  \bk{\vp}{\vr}  }_ {\bk{\vr}{\vp}^*} \bk{\vr}{\psi} \, d\vr =
\int \psi^*_{\vp}(\vr) \psi(\vr) \, d\vr
\end{equation}
Формула \eqref{eq:6_3_11} --- аналог \eqref{eq:2_1_5}. 

\begin{equation}
\label{eq:6_3_12}
\op{\vr} \ket{\psi} = \ket{\phi}
\end{equation}

\begin{equation}
\label{eq:6_3_13}
\begin{split}
	\left.\bk{\vr}{\phi} \equiv \phi(\vr)\right|_{\text{\eqref{eq:6_3_12}}} =
	\bfkh{\vr}{\op{\vr}}{\psi} = 
	\bfkh{\vr}{\op{\vr} \cdot \mathds{1}_{\vsr}}{\psi} =
	\bfkh{\vr}{\op{\vr}\int d\vsr}{\vr'} \bk{\vsr}{\psi} = \\ =
	\left.\int d\vsr \bfkh{\vr}{\op{\vr}}{\vsr} \psi(\vsr) \right|_{\text{\eqref{eq:6_3_4}}} =
	\int d\vsr \, \vsr \delta(\vr - \vsr) \psi(\vsr) = \vr \psi(\vr) = \phi(\vr)
\end{split}
\end{equation}
$$
\op{\vr}=\vr
$$

\begin{excr}
Следуя схеме \eqref{eq:6_3_13}, показать, что действие произвольной функции от оператора координаты $U(\op{\vr}) \equiv \op{U}(\vr)$ на волновую функцию $\psi(\vr)$ сводится к умножению $\psi(\vr)$ на вещественную функцию $U(\vr)$, т.е. что $U(\op{\vr})=U(\vr)$. \\
(Аналогия с \textsection 2 гл. 2: \eqref{eq:2_2_4} $\op{\vr}=\vr$  и \eqref{eq:2_2_5} $U(\op{\vr})=U(\vr)$)
\end{excr}

\begin{equation}
\label{eq:6_3_14}
\op{\vp} \ket{\psi} = \ket{\chi}
\end{equation}

\begin{equation}
\label{eq:6_3_15}
\begin{split}
	\left.\bk{\vp}{\chi} \equiv \chi(\vp)\right|_{\text{\eqref{eq:6_3_14}}} =
	\bfkh{\vp}{\op{\vp}}{\psi} = 
	\bfkh{\vp}{\op{\vp} \cdot \mathds{1}_{\vsp}}{\psi} =
	\bfkh{\vp}{\op{\vp}\int d\vsp}{\vp'} \bk{\vsp}{\psi} = \\ =
	\left.\int d\vsp \bfkh{\vp}{\op{\vp}}{\vsp} \psi(\vsr) \right|_{\text{\eqref{eq:6_3_3}}} =
	\int d\vsp \, \vsp \delta(\vp - \vsp) \psi(\vsp) = \vp \psi(\vp) = \chi(\vp)
\end{split}
\end{equation}
$$
\op{\vp}=\vp
$$

\begin{excr}
Следуя схеме \eqref{eq:6_3_15}, показать, что действие произвольной функции от оператора импульса $F(\op{\vp}) \equiv \op{F}(\vp)$ на волновую функцию $\psi(\vp)$ сводится к умножению, т.е. что $F(\op{\vp})=F(\vp)$, где $F(\vp)$ --- вещественная функция.
\end{excr}

Стационарное уравнение Шрёдингера в координатном ($\vr$-) представлении:
\begin{equation}
\label{eq:6_3_16}
\op{H}\ket{\psi} = \brc{\frac{\op{\vp}^2}{2m} + \op{U}(\vr)}\ket{\psi} = E\ket{\psi}
\end{equation}

$$
\bfk{\vp}{\op{H}}{\psi} = \bfk{\vp}{\op{H} \cdot \mathds{1}_{\vsp}}{\psi} =
\int d\vsp \, \bfk{\vp}{\op{H}}{\vsp} \bk{\vsp}{\psi} =
\int d\vsp \, \left[ \bfkh{\vp}{\op{T}}{\vsp} + \bfkh{\vp}{\op{U}(\vr)}{\vsp} \right] \psi(\vsp)
$$

% TODO: сделать нормальные ссылки на упражнения
\begin{equation}
\begin{split}
\label{eq:6_3_17}
	\left. \int d\vsp \, \bfkh{\vp}{\op{T}}{\vsp} \psi(\vsp) \right|_{\text{упр.2}} =
	\int d\vsp \, \bfkh{\vp}{ \frac{\vsp^2}{2m} }{\vsp} \psi(\vsp) = \\ =
	\int d\vsp \, \frac{\vsp^2}{2m} \underbrace{ \bkh{\vp}{\vsp} }_{\delta(\vp - \vsp)} \psi(\vsp) = \frac{\vp^2}{2m} \psi(\vp)
\end{split}
\end{equation}

\begin{equation}
\begin{split}
\label{eq:6_3_18}
	\bfkh{\vp}{\op{U}(\vr)}{\vsp} = \bfkh{\vp}{ \mathds{1}_\vr \cdot \op{U(\vr)} \cdot \mathds{1}_\vsr}{\vsp} =
	\iint d\vr d\vsr \, \underbrace{ \bk{\vp}{\vr} }_{\bk{\vr}{\vp}^*} \underbrace{ \bfkh{\vr}{\op{U}(\vr)}{\vsr} }_{U(\vsr)\delta(\vr - \vsr)} \bkh{\vsr}{\vsp} = \\ =
	\left. \int d\vr \, \psi^*_{\vp}(\vr) U(\vr) \psi_{\vsp}(\vr) \right|_{\text{\eqref{eq:6_3_9}}} =
	\boxed{ \frac{1}{(2\pi\hbar)^3} \int d\vr e^{-\frac{i}{\hbar} \brc{\vp - \vsp} \vr } U(\vr) = W(\vp - \vsp) }
\end{split}
\end{equation}
--- фурье-образ функции $U(\vr)$

Объединяя \eqref{eq:6_3_17} и \eqref{eq:6_3_18}, получим стационарное уравнение Шрёдингера в импульсном представлении (интегральное):
\begin{equation}
\label{eq:6_3_19}
	\boxed{\frac{p^2}{2m} \psi(\vp) + \int W(\vp - \vsp) \psi(\vsp) \, d\vp  = E\psi(\vp)}
\end{equation}

Из упражнения 1 2-го задания:
$$
\begin{gathered}
\op{\vr} = i\hbar \pd{}{\vp} \\
\op{\vp} = -i\hbar \pd{}{\vr}
\end{gathered}
$$

\section{Оператор эволюции. Представление Шрёдингера и Гайзенберга. Уравнение Гайзенберга для операторов физических величин}

В квантовой механике роль уравнений движения играют уравнения эволюции во времени.

\begin{equation}
\label{eq:6_4_1}
	i\hbar \pd{}{t} \ket{\psi(t)} = \op{H}\ket{\psi(t)}
\end{equation}

Введём новый оператор:
\begin{equation}
\label{eq:6_4_2}
	\ket{\psi(t)} = \op{U}(t)\ket{\psi(0)}
\end{equation}
$$
\bk{\psi(t)}{\psi(t)} = \bfk{\psi(0)}{\op{U}^\dag(t) \op{U}(t) }{\psi(0)} = \bk{\psi(0)}{\psi(0)}
$$
$$
\op{U}^\dag(t) \op{U}(t) = \mathds{1} ~~~ \text{(см. \textsection 2 гл.6)}
$$

$\op{U}$ --- оператор эволюции. Из вышеприведённой формулы видно, что он является унитарным.

Поставим \eqref{eq:6_4_2} в \eqref{eq:6_4_1}:
$$
\left\{ i\hbar \pd{}{t} \op{U}(t) - \op{H}\op{U}(t) \right\} \ket{\psi(0)} = 0 
$$
где $\ket{\psi(0)}$ --- любой ненулевой вектор-состояние.

\begin{equation}
\label{eq:6_4_3}
	i\hbar \pd{}{t} \op{U}(t) = \op{H}\op{U}(t)
\end{equation}

Положим, что $\pd{\op{H}}{t} = 0$, $\op{U}(0)=\mathds{1}$. Тогда:
\begin{equation}
\label{eq:6_4_4}
	\boxed{\op{U}(t)=e^{-\frac{i}{\hbar} \op{H}t}}
\end{equation}
--- экспоненциальный оператор (решение \eqref{eq:6_4_3})

\begin{defn}
\begin{equation}
\label{eq:6_4_5}
e^{-\frac{i}{\hbar} \op{H}t} \equiv \sum_{n=0}^{\infty} \frac{1}{n!} \brc{-\frac{i}{\hbar} \op{H}t}^n
\end{equation}
\end{defn}

\begin{defn}
Описание временной эволюции квантовой системы, когда вектор-состояние (или волновая функция) зависит от времени, а операторы не зависят от времени, называется \underbar{представлением Шрёдингера} \footnote{В некоторой литературе вместо термина <<представление>> употребляется термин <<картина>>}
\end{defn}

Обозначим вектор-состояние в представлении Гайзенберга через $\ket{\psi_H}$, а вектор-состояние в представлении Шрёдингера через $\ket{\psi_S}$
\begin{equation}
\label{eq:6_4_6}
	\left. \ket{\psi_H} \equiv \ket{\psi_S(0)} \right|_\text{\eqref{eq:6_4_2}} \equiv 
	\left.  \op{U}^\dag(t) \ket{\psi_S(t)} \right|_\text{\eqref{eq:6_4_4}} =
	e^{\frac{i}{\hbar}\op{H}_St} \ket{\psi_S(t)}
\end{equation}

\begin{excr}
Доказать, что если $\op{U}(t)=\exp\brc{-\frac{i}{\hbar}\op{H}t}$, то $\op{U}^\dag(t) = \exp\brc{\frac{i}{\hbar}\op{H}t}$, используя явное определение \eqref{eq:6_4_5}.
\end{excr}

Согласно \eqref{eq:6_2_2} и \eqref{eq:6_2_4_add}:
\begin{equation}
\label{eq:6_4_7}
\boxed{
	\op{F_H}(t) = \op{U}^\dag(t) \op{F_S} \op{U}(t) = e^{(i/\hbar) \op{H_S}t} \op{F_S} e^{-(i/\hbar) \op{H_S}t}
}
\end{equation}

\begin{excr}
Доказать:
$$
e^{\xi \op{A}} \op{B} e^{-\xi \op{A}} = \op{B} + \xi \brs{\op{A}, \op{B}} + \frac{\xi^2}{2!} \brs{\op{A}, \brs{\op{A}, \op{B}}} + ...
$$
где $\xi = \frac{i}{\hbar} t$, $\op{A} = \op{H_S}$, $\op{B} = \op{F_S}$
\end{excr}

\begin{equation}
\label{eq:6_4_8}
\op{F_H}(t) = \op{F_S} + \frac{i}{\hbar} t \brs{\op{H_S}, \op{F_S}} - \frac{t^2}{2\hbar^2} \brs{\op{H_S}, \brs{\op{H_S}, \op{F_S}}} + ...
\end{equation}

Если $\brs{\op{H_S}, \op{F_S}} = 0$, т.е. (см \S 2 гл. 5) $\op{F_S}$ является оператором интеграла движения, то
$$
\op{F_H}(t) = \op{F_S}
$$
т.е. гайзенберговские операторы интегралов движения не зависят от времени и совпадают с соответствующими в представлении Шрёдингера.

Гамильтониан в обоих представлениях совпадает и не зависит от времени:
$$
\op{H_H} = \op{H_S} = \op{H}
$$

Из \eqref{eq:6_4_7}:
\begin{equation}
\label{eq:6_4_9}
\op{F_H}(t) = e^{(i/\hbar)\op{H}t} \op{F_S} e^{-(i/\hbar)\op{H}t}
\end{equation}

Уравнение движения для гайзенберговского оператора $\op{F_H}(t)$
\begin{equation}
\label{eq:6_4_10}
\boxed{
	\D{}{t}\op{F_H}(t) = \frac{i}{\hbar} \brs{\op{H}, \op{F_H}}
}
\end{equation}

\begin{defn}
Представление, в котором эволюция во времени переновится на операторы, а векторы-состояния от времени не зависят, называется представлением Гайзерберга
\end{defn}

Из \eqref{eq:5_2_6}:
$$
\D{\op{F}}{t} = \pd{\op{F}}{t} + \frac{i}{\hbar} \brs{\op{H}, \op{F}}
$$

Различные представления \textbf{унитарно-эквивалентны}, т.к. дают эквивалентные результаты, и переход между ними осуществляется с помощью унитарных преобразований. Таким образом, одна и та же задача может решаться проще в одном из представлений.
