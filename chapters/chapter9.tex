\chapter{Движение в центрально-симметричном поле}

\begin{sloppypar}
\section{Центрально-симметричное поле. Гамильтониан частицы в сферических координатах. Разделение переменных в центрально-симметричном поле.}
\end{sloppypar}

Если поле центрально-симметричное, т.е. потенциал сферически симметричен, то $U(\vr) \equiv U(r)$. Ввиду сферической симметрии поля, задачу о движении частицы в нём удобно решать в сферической системе координат $(r, \theta, \phi)$, начало которой совпадает с центром симметрии поля. Тогда гамильтониан движения частицы в сферических координатах, согласно \eqref{eq:8_4_6}, принимает вид

\begin{equation}
\label{eq:9_1_1}
\op{H} = \frac{\op{\vp}^2}{2m} + U(r) = - \frac{\hbar^2}{2m} \brs{\frac{1}{r^2} \pd{}{r} \brc{r^2 \pd{}{r}} - \frac{\op{\vec{l}}^2}{r^2}} + U(r)
\end{equation}%
%
Знание интегралов движения системы позволяет упростить решение уравнения Шрёдингера, поэтому в случае центрально-симметричного поля необходимо выявить сначала сохраняющиеся физические величины. Можно показать, что $\brs{\op{H}, \op{l}_\alpha} = 0$ (см.~\cref{ex:1_2_5}) и $\brs{\op{H}, \op{\vec{l}}^2} = 0$ (см.~\cref{ex:1_1_6}). Кроме того, $\brs{\op{\vec{l}^2}, \op{l}_\alpha} = 0$ (см.~\eqref{eq:8_4_5}). Таким образом, операторы $\op{H}$, $\op{\vec{l}}^2$ и $\op{l}_\alpha$ порождают полный набор наблюдаемых, пригодный для описания движения бесспиновой частицы в центрально-симметричном поле. Они имеют общую систему собственных функций, описывающих стационарные состояния движения частицы в сферически симметричном поле:

\begin{equation}
\label{eq:9_1_2}
\bk{\vec{r}}{nlm} \equiv \psi_{nlm}(\vr) = R_{nl}(r) \cdot Y_{lm}(\theta, \phi)
\end{equation}%
%
где $n$ -- {\em главное}, $l$ -- {\em орбитальное}, $m$ -- {\em магнитное} квантовые числа. Структура гамильтониана \eqref{eq:9_1_1} такова, что радиальные и угловые переменные в решении \eqref{eq:9_1_2} уравнения Шрёдингера разделяются.

\section{Уравнение для радиальной функции}

Будем искать решения стационарного трёхмерного уравнения Шрёдингера

$$
-\frac{\hbar^2}{2m} \brs{ \frac{1}{r^2} \pd{}{r}\brc{r^2 \pd{}{r}} - \frac{\op{\vec{l}}^2}{r^2}} \psi(r, \theta, \phi) + U(r)\psi(r, \theta, \phi) = E \psi(r, \theta, \phi)
$$%
%
в виде \eqref{eq:9_1_2}, т.е.
$$
\begin{gathered}
\psi(r, \theta, \phi) = R_{nl}(r) \underbrace{Y_{lm}(\theta, \phi)}_{\mathclap{\text{сферические гармоники (функции)}}}\\
\text{где}~\op{\vec{l}}^2 Y_{lm}(\theta, \phi) = l (l+1) Y_{lm}(\theta, \phi)~\text{см.~\eqref{eq:8_5_3}}
\end{gathered}
$$

Подставляя и сокращая на $Y_{lm}(\theta, \phi)$, получим
$$
-\frac{\hbar^2}{2m} \brs{ \frac{1}{r^2} \D{}{r} \brc{r^2 \D{}{r} R_{nl}(r) } - \frac{l(l+1)}{r^2} R_{nl}(r) } + U(r) R_{nl}(r) = E R_{nl}(r)
$$%
%
После небольшой перегруппировки приходим к уравнению для радиальной части волновой функции
\begin{equation}
\label{eq:9_2_1}
\boxed {
	\frac{1}{r^2} \pd{}{r} \brc{r^2 \D{}{r} R_{nl}(r)} - \frac{l(l+1)}{r^2} R_{nl}(r) + \frac{2m}{\hbar^2} \brc{E - U(r)} R_{nl}(r) = 0
}
\end{equation}

Таким образом, за счёт разделения \eqref{eq:9_1_2} угловых и радиальных переменных от трёхмерного уравнения мы перешли в случае центрально-сим\-мет\-рич\-но\-го поля к одномерному уравнению Шрёдингера (уравнению для радиальной функции). Это уравнение \eqref{eq:9_2_1} следует рассматривать как стартовое для решения задач 2б ($U(r) = m \omega^2 r^2 / 2$), 6 и 7* из 2-го задания.

Заметим, что волновая функция должна быть нормирована на единицу. В нашем случае условие нормировки может быть записано следующим образом

$$
\begin{gathered}
\int \abs{\psi_{nlm}(\vr)}^2 \diff\vr
  = \iiint \abs{R_{nl}(r)}^2 \cdot \abs{Y_{lm}(\theta, \phi)}^2
      r^2 \diff r \underbrace{\diff\Omega}_{\mathclap{\sin\theta \diff\theta \diff\phi}} = \\
  = \int_{0}^{\infty} \abs{R_{nl}(r)}^2
      r^2 \diff r \underbrace{\oint \abs{Y_{lm}(\theta, \phi)}^2 \diff\Omega}_{
        =1 \text{ из \eqref{eq:8_5_4}}
      } = 1
\end{gathered}
$$

Следовательно, радиальная часть волновой функции $R_{nl}(r)$ удовлетворяет условию нормировки

\begin{equation}
\label{eq:9_2_2}
\boxed {
	\int_{0}^{\infty} \abs{R_{nl}(r)}^2 r^2 \diff r = 1
}
\end{equation}
