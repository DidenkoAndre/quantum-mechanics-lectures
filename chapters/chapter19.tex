\chapter{Теория рассеяния}

\section{Постановка задачи рассеяния. Упругое рассеяние}

\begin{figure}[h!]
\centering
\begin{tikzpicture}[domain=-5:5]
  \draw[->] (-5, 0) -- (5, 0) node[above] {$z$};
  \draw[thick, ->] (-5, 0) -- (-4, 0) node[above] {$\vp$};
  \draw[->] (0, 0) -- (3,3);
  \draw[thick, ->] (2.3, 2.3) -- (3,3) node [below] {$\vsp$};
  \draw[very thick, -] (2.8, 3.3) -- (3.6,3.3) ;
  \draw[very thick, -] (3.6, 3.3) -- (3.3,3) ;
  \draw[-] (0, 0) -- (2.8,2.4);
  \draw[-] (0, 0) -- (2.4,2.8);
  \node [right] at (1.7, 2.9) {$d\Omega$} ;
  \node at (-3, 2) {$U(\vr) \neq 0,$} ;
  \node at (-3, 1.5) {$ \text{при } r<a$};
  \draw [rotate = -45] (0, 3.75) ellipse (3mm and 2 mm) ;
  \draw [dashed] (0, 0) circle (1cm);
  \draw [->] (0, 0) -- (-0.7, 0.7) node [left] {$a$}; 
  \draw [dashed] (0.7, 0) arc (0:45:7mm) ;
  \node [right] at (0.6, 0.3) {$\theta$};
  \draw [blue, dashed] (-3, 0) .. controls (0, 0) .. (3, 3);
\end{tikzpicture}
\caption{Постановка задачи рассеяния.} \label{fig:19_1}
\end{figure}

$$
U(\vr) \equiv 0, ~~~ r > a
$$

На детектор приходит рассеянная волна.

\begin{equation}
\label{eq:19_1_1}
\op{H} \psi(\vr) = E \psi(\vr),
\end{equation}

где $\op{H} = \frac{\op{\vp}^2}{2m} + U(\vr)$.


При $E < 0$ задача превращается в задачу на собственные значения, решение которой мы знаем.

$$
E = \frac{\hbar^2 k^2}{2m} > 0
$$

Запишем определение упругого рассеяния:
\begin{gather*}
\frac{\op{\vp}^2}{2m} = \frac{\op{\vsp}^2}{2m} \to \abs{\vp} = \abs{\vsp} \\
\vp \not = \vsp
\end{gather*}

Рассмотрим асимптотику волновой функции вдали от рассеивающего центра:
$$
\psi(\vr) \left|_{r \to \infty} \approx e^{i \vec{k} \vr} + f(\theta, \phi) \frac{e^{ikr}}{r} \right . 
$$

Функция $f(\theta, \phi)$ - амплитуда рассеяния.

\begin{sloppypar}
Если рассеивающий потенциал сферически-симметричен, т.е. $U(\vr) = U(r)$, то $f(\theta, \phi) = f(\theta)$.
\end{sloppypar}

\begin{equation}
\label{eq:19_1_2}
\psi(\vr) \left|_{r \to \infty} \approx e^{i \vec{k} \vec{\vr}} + f(\theta) \frac{e^{ikr}}{r} \right .
\end{equation}

Волновая функция, удовлетворяющая \eqref{eq:19_1_2} называется функцией рассеяния.

\section{Амплитуда и сечение рассеяния}

Пусть $\D{N}{t}$ - скорость счета детектора. В детектор попадают частицы из телесного угла $d\Omega$.

\begin{defn}
Плотность потока падающих частиц $\vec{j}_{\text{пад}}$ - число частиц, проходящих в единицу времени через единичную площадку, перпендикулярную пучку.
\end{defn}

\begin{defn}
Эффективным сечением рассеяния (сечением рассеяния) называется отношение:
\begin{equation}
\label{eq:19_2_1}
\frac{dN}{dt \abs{\vec{j}_{\text{пад}}}} = d \sigma
\end{equation}
\end{defn}

Заметим, что размерность $d\sigma = [\text{см}^2]$.

С другой стороны:
$$
\D{N}{t} = j_{\text{рас}}r^2 d\Omega
$$

Тогда 
\begin{equation}
\label{eq:19_2_2}
d\sigma = \frac{j_{\text{рас}}}{j_{\text{пад}}}r^2 d\Omega
\end{equation}

Из \S 1 гл. 5:
$$
j_{\text{пад}} \sim J: ~~~\vec{J} = \frac{-i\hbar}{2m}\brc{\psi^* \nabla \psi - \psi \nabla \psi^*}
$$

$$
\rho_{\text{рас}} = \frac{\abs{f(\theta)}^2}{r^2}
$$
$$
j_{\text{рас}} = \rho_{\text{рас}} v = v \frac{\abs{f(\theta)}^2}{r^2}
$$

Подставим это выражение в \eqref{eq:19_2_2}:
\begin{equation}
\label{eq:19_2_3}
\D{\sigma}{\Omega} = \abs{f(\theta)}^2
\end{equation}

\begin{sloppypar}
- выражение для дифференциального сечения рассеяния в сферически-симметричном рассеивающем потенциале.
\end{sloppypar}

\section{Функция Грина задачи рассеяния. Интегральное уравнение задачи рассеяния}

Запишем уравнение Шредингера для задачи рассеяния:
\begin{gather*}
\brc{-\frac{\hbar^2}{2m} \Delta + U(\vr)}\psi(\vr) = \frac{\hbar^2 k^2}{2m} \psi(\vr) ~~~| \cdot \frac{2m}{\hbar^2} \\
(\Delta + k^2) \psi(\vr) = \frac{2m}{\hbar^2}U(\vr)\psi(\vr)
\end{gather*}

Решение этого неоднородного дифференциального уравнения можно представить, как сумму общего решения однородного ДУ и частного решения неоднородного ДУ:
$$
\left \{  
\begin{matrix}
(\Delta + k^2) \psi_0(\vr) = 0\\
(\Delta + k^2) \psi_1(\vr) = \frac{2m}{\hbar^2}U(\vr)\psi(\vr),
\end{matrix}
\right .
$$

Вспомним определение функции Грина $G(\vr - \vsr)$:
\begin{equation}
\label{eq:19_3_1}
(\Delta_{\vr} + k^2)G(\vr - \vsr) \equiv \delta(\vr - \vsr)
\end{equation}

Тогда с использованием функции Грина:
$$
\psi_1(\vr) = \int G(\vr - \vsr) \frac{2mU(\vsr)}{\hbar^2} \psi(\vsr) d\vsr
$$

$$
G(\vr - \vsr) = - \frac{e^{ik\abs{\vr - \vsr}}}{4\pi\abs{\vr - \vsr}}
$$

Теперь можно выписать функцию рассеяния:
$$
\psi(\vr) = \psi_0(\vr) - \frac{1}{4 \pi}\frac{2m}{\hbar^2}\int \frac{e^{ik\abs{\vr - \vsr}}}{\abs{\vr - \vsr}} U(\vsr)\psi(\vsr)d\vsr
$$

Всегда можно считать, что $|\vsr| < a \ll r$, т.к. $r \to \infty$.

Введем малый параметр:
$$
\frac{r'}{r} \ll 1
$$

Тогда:
$$
\frac{e^{ik\abs{\vr - \vsr}}}{\abs{\vr - \vsr}} \approx \frac{e^{ikr - ik\vsr \vec{n}}}{r},
$$

где $\vec{n} = \frac{\vr}{r}$.

$$
\psi(\vr) = \psi_0(\vr) - \frac{e^{ikr}}{r}\frac{m}{2\pi\hbar^2}\int e^{- ik\vsr \vec{n}} U(\vsr)\psi(\vsr)d\vsr
$$

С учетом $\psi_0(\vr) = e^{i\vec{k}\vr}$:

$$
f(\theta, \phi) =  - \frac{m}{2\pi\hbar^2}\int e^{- ik\vsr \vec{n}} U(\vsr)\psi(\vsr)d\vsr
$$

Окончально получим интегральное уравнение задачи рассеяния:

\begin{equation}
\label{eq:19_3_2}
\boxed{\psi(\vr) = e^{i\vec{k}\vr} - \frac{m}{2\pi\hbar^2}\int \frac{e^{ik\abs{\vr - \vsr}}}{\abs{\vr - \vsr}} U(\vsr)\psi(\vsr)d\vsr}
\end{equation}

Окончание лекции см. \href{http://mipt.ru/education/chair/theoretical_physics/upload/1e2/Born_4-arpg3lk8s2a.pdf}{тут}. Текст ссылки: \\
mipt.ru/education/chair/theoretical\_physics/upload/1e2/Born\_4-arpg3lk8s2a.pdf
