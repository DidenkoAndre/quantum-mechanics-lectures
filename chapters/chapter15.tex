\chapter{Сложение моментов}

\section{Сложение моментов}

Пусть имеются две системы с угловыми моментами $j_1$ и $j_2$. Каждая из систем определена в собственном пространстве, т.е. $\mathcal{H} = \mathcal{H}^{(1)} \otimes \mathcal{H}^{(2)}$, поэтому

$[\op{j}_{1 \alpha}, \op{j}_{2 \beta}] = 0$ при $\forall \alpha, \beta = x, y, z$.

Состояние первой системы определяется собственными векторами $\ket{j_1 m_1}$ операторов $\op{\vec{j}}_1^2$ и $\op{j}_{1z}$, т.е.
\begin{gather*}
\op{\vec{j}}_1^2\ket{j_1 m_1} = j_1(j_1 + 1)\ket{j_1 m_1},  \\
\op{j}_{1z} \ket{j_1 m_1} = m_1\ket{j_1 m_1},
\end{gather*}
где $m_1 = -j_1, -j_1 + 1, \dots , j_1 - 1, j_1$ (см. \S 2 главы \rom{8}). Аналогично для второй системы имеем:
\begin{gather*}
\op{\vec{j}}_2^2\ket{j_2 m_2} = j_2(j_2 + 1)\ket{j_2 m_2},  \\
\op{j}_{2z} \ket{j_2 m_2} = m_2\ket{j_2 m_2}, 
\end{gather*}
где $m_2 = -j_2, -j_2 + 1, \dots , j_2 - 1, j_2$.

Объединенная система <<$1 + 2$>> характеризуется набором коммутирующих операторов $\op{\vec{j}}_1^2$, $\op{j}_{1z}$, $\op{\vec{j}}_2^2$ и $\op{j}_{2z}$. Соответственно состояния объединенной системы описываются векторами:
$$
\ket{j_1 m_1 j_2 m_2} \equiv \ket{j_1 m_1} \ket{j_2 m_2} 
$$

Их количество $(2j_1 + 1)(2j_2 + 1)$ совпадает с размерностью прямого произведения пространств $\mathcal{H}^{(1)} \otimes \mathcal{H}^{(2)} = \mathcal{H}$.

Введем оператор полного углового момента систем 1 и 2:
$$
\op{\vec{j}} = \op{\vec{j}}_1 + \op{\vec{j}}_2
$$

Возникает новый набор коммутирующих операторов: $\op{\vec{j}}_1^2$, $\op{\vec{j}}_2^2$, $\op{\vec{j}}^2$ и $\op{j}_{z}$.
\begin{excr}
Доказать, что $\op{\vec{j}}_1^2$, $\op{\vec{j}}_2^2$, $\op{\vec{j}}^2$ и $\op{j}_{z}$ взаимно коммутируют.
\end{excr}

Пусть $\brcr{\ket{j m j_1 j_2}}$ --- набор общих собственных векторов этих операторов, т.е.
\begin{gather*}
\op{\vec{j}}_1^2\ket{j m j_1 j_2} = \lambda_1 \ket{j m j_1 j_2}  \\
\op{\vec{j}}_2^2\ket{j m j_1 j_2} = \lambda_2 \ket{j m j_1 j_2}  \\
\op{\vec{j}}^2\ket{j m j_1 j_2} = j(j+1) \ket{j m j_1 j_2}  \\
\op{j}_{z} \ket{j m j_1 j_2} = m \ket{j m j_1 j_2}, 
\end{gather*}
где собственные значения $\lambda_1$, $\lambda_2$, $j(j+1)$, $m$ подлежат определению. На данном этапе мы можем быть уверены только в том, что $j$ --- это, как любой угловой момент, целое или полуцелое число, и что при фиксированном $j$ проекция $m$ пробегает значения $-j, -j + 1, \dots , j - 1, j$.

Задача сложения угловых моментов состоит в том, чтобы по известным $j_1$ и $j_2$, а также известным $\ket{j_1 m_1}$ и $\ket{j_2 m_2}$ установить, во-первых, значения $j$ (и, конечно, $\lambda_1$ и $\lambda_2$) и, во-вторых, вид векторов состояний $\ket{j m j_1 j_2} \equiv \ket{j m}$ (известные значения $j_1$, $j_2$ обычно опускают при записи собственных векторов операторов $\op{\vec{j}}^2$ и $\op{j}_{z}$).

\begin{sloppypar}
\section{Коэффициенты Клебша-Гордана. Полный угловой момент}
\end{sloppypar}

Неизвестные векторы состояний $\ket{j m j_1 j_2}$ всегда могут быть представлены в виде разложения по известному базису $\ket{j_1 m_1} \ket{j_2 m_2}$:
\begin{equation}
\label{eq:15_2_1}
\boxed{\ket{j m j_1 j_2} = \sum_{m_1, m_2} C^{j m}_{j_1 m_1 j_2 m_2} \ket{j_1 m_1}\ket{j_2 m_2}}
\end{equation}

Коэффициенты разложения $C^{j m}_{j_1 m_1 j_2 m_2}$ называют коэффициентами Клебша-Гордана. Действуя на эти разложения \eqref{eq:15_2_1} операторами $\op{\vec{j}}_1^2$ и $\op{\vec{j}}_2^2$, легко находим их собственные значения:
\begin{gather*}
\lambda_1 = j_1(j_1 + 1) \\
\lambda_2 = j_2(j_2 + 1)
\end{gather*}

Дальнейшее исследование разобьем на пункты.

1) Воспользуемся тем, что $\op{j}_z = \op{j}_{1z} + \op{j}_{2z}$. Действуя оператором $\op{j}_z$ на левую часть \eqref{eq:15_2_1}, а оператором $\op{j}_{1z} + \op{j}_{2z}$ на его правую часть, получим:
\begin{equation}
\label{eq:15_2_2}
m\ket{j m} = \sum_{m_1, m_2} C^{j m}_{j_1 m_1 j_2 m_2} (m_1 + m_2)\ket{j_1 m_1}\ket{j_2 m_2}
\end{equation}

Из сравнения \eqref{eq:15_2_2} с \eqref{eq:15_2_1} следует, что $C^{j m}_{j_1 m_1 j_2 m_2} = 0$, если $m \neq m_1 + m_2$ или
$$
\boxed{C^{j m}_{j_1 m_1 j_2 m_2} \neq 0, \text{~только если~} m = m_1 + m_2}
$$

Последнее соотношение означает, что в разложениях \eqref{eq:15_2_1} индексы суммирования $m_1$ и $m_2$ не являются независимыми, т.е. суммирование реально проводится только по одному индексу:
\begin{equation}
\label{eq:15_2_3}
\ket{j m} = \sum_{m_1} C^{j m}_{j_1 m_1 j_2 (m - m_1)} \ket{j_1 m_1}\ket{j_2, m - m_1}
\end{equation}

2) С одной стороны, $m_{max} = j_{max}$. С другой стороны, $m_{max} = m_{1max} + m_{2max} = j_1 + j_2$. Следовательно, $\boxed{j_{max} = j_1 + j_2}$.

Легко понять, что при $j = j_{max}$ и $m = m_{max} = j_{max}$, разложение \eqref{eq:15_2_1} (или \eqref{eq:15_2_3}) сводится к единственному слагаемому, в котором $m_1 = m_{1max} = j_1$ и $m_2 = m_{2max} = j_2$, т.е.
$$
\ket{j_{max}~j_{max}} = C^{j_1+j_2, j_1 + j_2}_{j_1~j_1~j_2~j_2} \ket{j_1~j_1}\ket{j_2~j_2}
$$

Откуда из условия нормировки $\bk{j_{max}~j_{max}}{j_{max}~j_{max}} = 1$ следует, что 
$$
\boxed{C^{j_1+j_2, j_1 + j_2}_{j_1~j_1~j_2~j_2} = 1}
$$

3) Установим теперь значение $j_{min}$. Следуя векторной модели сложения векторов $\vec j = \vec j_1 + \vec j_2$ (см.~\llref{31}{3}), естественно предположить, что 
$$
\boxed{j_{min} = \abs{j_1 - j_2}}
$$

Докажем это следующим образом. Пусть $j_1 \ge j_2$ и $j_{min} = j_1 - j_2$. Тогда общее количество собственных векторов $\ket{j m}$ определяется суммой:
$$
\sum_{j = j_1 - j_2}^{j_1 + j_2} (2j + 1) = 2 \sum_{j = j_1 - j_2}^{j_1 + j_2} j + \sum_{j = j_1 - j_2}^{j_1 + j_2} 1 = (2j_1 + 1)(2j_2 + 1)
$$
как и должно быть из прямого произведения пространств $\mathcal{H}^{(1)} \otimes \mathcal{H}^{(2)} = \mathcal{H}$.
 
Итак, мы установили, что собственные векторы $\ket{j m j_1 j_2} \equiv \ket{j m}$ операторов $\op{\vec{j}}^2$, $\op{\vec{j}}_1^2$, $\op{\vec{j}}_2^2$ и $\op{j}_z$ определяется разложениями \eqref{eq:15_2_1}. При этом число $j$ (полный угловой момент объединенной системы) меняется через единицу от $\abs{j_1 - j_2}$ до $j_1 + j_2$, т.е. возможными значениями $j$ являются:
\begin{equation}
\label{eq:15_2_4}
\boxed{j = \abs{j_1 - j_2}, \abs{j_1 - j_2} + 1, \dots, j_1 + j_2}
\end{equation}

Для каждого фиксированного $j$ проекция $m$ полного углового момента на ось $z$ принимает значения $-j, -j + 1, \dots, j -1, j$. Таким образом, задача сложения угловых моментов сводится к определению численных значений коэффициентов Клебша-Гордана. Одно из этих значений нам уже известно: $C^{j_1+j_2, j_1 + j_2}_{j_1~j_1~j_2~j_2} = 1$. Для установления остальных значений может быть применен способ, использующий оператор понижения
\begin{gather*}
\op{j}_- = \op{j}_x - i \op{j}_y = \op{j}_{1x} + \op{j}_{2x} - i (\op{j}_{1y} + \op{j}_{2y}) = \op{j}_{1-} + \op{j}_{2-}
\end{gather*}