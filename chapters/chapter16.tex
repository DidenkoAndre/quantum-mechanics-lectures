\chapter{Тождественные частицы}

\begin{sloppypar}
\section{Симметрия волновой функции системы тождественных частиц. Бозоны и фермионы}
\end{sloppypar}

В классической механике описание движения системы одинаковых частиц сводится к определению закона движения каждой частицы в отдельности. При этом одинаковые частицы, несмотря на тождественность их физических свойств, не теряют своей <<индивидуальности>>: за движением каждой из них можно, в принципе, проследить по ее траектории (см. \autoref{fig:16_1}a).

\begin{figure}[h!]
\centering
\begin{tikzpicture}[scale=0.7][domain=-10:10]
  \draw (-8, 0) circle (1mm) node[left] {$\vr_2$};
  \draw (-8, 2) circle (1mm) node[left] {$\vr_1$};
  \node[below] at (-8, -0.5) {$t=0$};
  \node[below] at (-5, -0.5) {$t>0$};
  \node[below] at (-6.5, -1) {$a)$};    
  \draw (-5, 0.7) circle (1mm);
  \draw (-5, 3) circle (1mm);
  \draw[shorten <= 3, shorten >= 4] (-8, 0) arc (110:95:12cm); 
  \draw[shorten <= 3, shorten >= 4] (-8, 2) arc (120:97:8cm);  
  \draw[shorten <= 3, shorten >= 4] (-5, 0.7) arc (95:85:12cm); 
  \draw[shorten <= 3, shorten >= 4] (-5, 3) arc (97:82:8cm);

  \draw (0, 0) circle (1mm) node[left] {$\vr_2$};
  \draw (0, 2) circle (1mm) node[left] {$\vr_1$};
  \node[below] at (0, -0.5) {$t=0$};
  \node[below] at (3, -0.5) {$t>0$};
  \node[below] at (1.5, -1) {$\text{б})$}; 
  \draw[shorten <= 3, shorten >= 4] (0, 0) arc (130:108:12cm); 
  \draw[shorten <= 3, shorten >= 4] (0, 2) arc (130:97:8cm);
  \draw[shorten <= 3, shorten >= 4] (0, 0) arc (90:70:12cm);
  \draw[shorten <= 3, shorten >= 4] (0, 2) arc (90:61:8cm);
  \node[above] at (5.5, 3.5) {\text{Расплывание}};
  \node[above] at (5.5, 3) {\text{волновых}};
  \node[above] at (5.5, 2.5) {\text{пакетов}};    
  \node[above] at (5.5, 2) {\text{частиц}}; 
  \draw[->] (5, 2) -- (3.5, 0.2);
  \draw[->] (5, 2) -- (3.5, 2.5);
  \draw[-] (2.6, 1.6) -- (2.65, 1.7);
  \draw[-] (2.7, 1.55) -- (2.8, 1.75);
  \draw[-] (2.8, 1.50) -- (2.95, 1.8);
  \draw[-] (2.9, 1.45) -- (3.1, 1.85);
  \draw[-] (3, 1.4) -- (3.25, 1.9);
  \draw[-] (3.1, 1.35) -- (3.4, 1.95);
  \draw[-] (3.2, 1.3) -- (3.55, 2);
  \draw[-] (3.32, 1.25) -- (3.72, 2.05);
  \draw[-] (3.45, 1.2) -- (3.85, 2);
  \draw[-] (3.57, 1.15) -- (3.92, 1.85);
  \draw[-] (3.7, 1.1) -- (4, 1.7);
\end{tikzpicture}
\caption{Траектории частиц в классической (а) и квантовой (б) механике.} \label{fig:16_1}
\end{figure}

В квантовой механике качественно иная ситуация. Здесь принципиально нет никакой возможности следить в отдельности за каждой из одинаковых частиц (хотя бы потому, что понятие траектории теряет смысл) и тем самым различать их (см. \autoref{fig:16_1}б).

В квантовой механике одинаковые частицы полностью теряют свою <<индивидуальность>>. Имеет смысл характеризовать систему из $N$ тождественных частиц общей $N$-частичной волновой функцией $\Psis(\xi_1, \xi_2, \dots, \xi_N )$, где $\xi_i = (\vr_i, \sigma_i)$ --- совокупность пространственных и спиновых переменных $i$-й частицы.

Главная особенность квантовомеханического описания вытекает из особого физического свойства микрообъектов, известного как {\em принцип тождественности или неразличимости микрочастиц}. Согласно этому принципу, \underline{состояние системы тождественных частиц не меняется при обмене частиц местами}. Иными словами, если $\op{P}$ --- оператор перестановки двух произвольных тождественных частиц (от англ. <<permutation>> --- перестановка) и
$$
\op{P} \Psis(\xi_1, \xi_2, \dots, \xi_N) \equiv \Psis(\xi_2, \xi_1, \dots, \xi_N),
$$
то волновые функции $\Psis(\xi_1, \xi_2, \dots, \xi_N)$ и $\Psis(\xi_2, \xi_1, \dots, \xi_N)$ описывают одно и то же состояние $N$-частичной системы. Найдем собственные значения $P$ оператора $\op{P}$:
\begin{equation}
\label{eq:16_1_1}
\op{P} \Psis(\xi_1, \xi_2, \dots, \xi_N) \equiv \Psis(\xi_2, \xi_1, \dots, \xi_N) = P \Psis(\xi_1, \xi_2, \dots, \xi_N)
\end{equation}

С другой стороны, из \eqref{eq:16_1_1} следует, что:

\begin{gather*}
\op{P}^2 \Psis(\xi_1, \xi_2, \dots, \xi_N) = \op{P}\Psis(\xi_2, \xi_1, \dots, \xi_N) = \Psis(\xi_1, \xi_2, \dots, \xi_N) = \\= \op{P} P\Psis(\xi_1, \xi_2, \dots, \xi_N) = P^2 \Psis(\xi_1, \xi_2, \dots, \xi_N)
\end{gather*}

Отсюда $P^2 = 1$ или собственные значения оператора перестановки тождественных частиц 
$$\boxed{P = \pm 1}$$

При $P = +1$
$$
\Psis(\xi_1, \xi_2, \dots, \xi_N) = \Psis(\xi_2, \xi_1, \dots, \xi_N)
$$
--- {\em симметричная волновая функция}.

При $P = -1$
$$
\Psis(\xi_1, \xi_2, \dots, \xi_N) = -\Psis(\xi_2, \xi_1, \dots, \xi_N)
$$ 
--- {\em антисимметричная волновая функция}.

Симметрия волновой функции относительно перестановки тождественных частиц однозначно связана со спином этих частиц (постулат или закон природы, установленный экспериментально):

а) {\em бозоны} (статистика Бозе-Эйнштейна\footnotemark{}) описываются симметричной волновой функцией --- частицы с целым спином (включая $s = 0$).
%people
\footnotetext{Сатьендра Нат Б\'{о}зе (Satyendra Nath Bose, 1894-1974)}

б) {\em фермионы} (статистика Ферми-Дирака) описываются антисимметричной волновой функцией --- частицы с полуцелым спином.

\section{Детерминант Слэтера. Принцип Паули}

Рассмотрим систему из $N$ тождественных фермионов в приближении их слабого взаимодействия между собой, которое учитывается по теории возмущений. Тогда гамильтониан в нулевом порядке ТВ аддитивен:
$$
\op{H} = \sum_{i=1}^N\op{H}_i
$$

Пусть $\brcr{\psi_{n_i}(\xi_i)}$ --- полная система собственных функций для всех $\op{H}_i$, где $n_i$ --- мультииндекс. {\em Мультииндекс} --- это полный набор квантовых чисел $i$-го фермиона. Тогда $N$-частичная волновая функция мультипликативна:
\begin{equation}
\label{eq:16_2_1}
\Psis(\xi_1, \dots, \xi_N) = \psi_{n_1}(\xi_1) \cdot \dots \cdot \psi_{n_N}(\xi_N)
\end{equation}

Полная нормированная волновая функция системы $N$ фермионов записывается как антисимметричная комбинация произведений вида \eqref{eq:16_2_1} с учетом всевозможных перестановок внутри таких произведений:
\begin{equation}
\label{eq:16_2_2}
\Psis^A(\xi_1, \dots, \xi_N) = \frac{1}{\sqrt{N!}} \left |
  \begin{matrix} 
  \psi_{n_1}(\xi_1) & \psi_{n_2}(\xi_1) & \dots &  \psi_{n_N}(\xi_1) \\
  \psi_{n_1}(\xi_2) & \psi_{n_2}(\xi_2) & \dots &  \psi_{n_N}(\xi_2) \\
  .                                &                                  &    &                                    \\
  .                                &                                  &    &                                    \\
  .                                &                                  &    &                                    \\
  \psi_{n_1}(\xi_N) & \psi_{n_2}(\xi_N) & \dots &  \psi_{n_N}(\xi_N)
  \end{matrix} \right |
\end{equation}
--- {\em детерминант Слэтера}\footnotemark{}.
%people
\footnotetext{Джон Кларк Слэтер (John Clarke Slater, 1900-1976)}
%
\begin{excr}

Из условия ортонормировки одночастичных волновых функций: 
$$
\int \psi_{n_i}^*(\xi_k) \psi_{n_j}(\xi_k) \diff\xi_k = \delta_{n_i n_j}
$$
и 
$$
\bk{\psi_n}{\psi_n} = 1
$$
получить нормировочный множитель в \eqref{eq:16_2_2}.
\end{excr}

Антисимметричный характер такой волновой функции очевиден, т.~к. перестановка двух частиц здесь соответствует перестановке двух строк определителя, в результате чего последний меняет знак.

Если среди номеров $n_1, n_2, \dots, n_N$ есть хотя бы два одинаковых, то два столбца определителя окажутся одинаковыми и весь определитель обратится в нуль. Он будет отличен от нуля только в тех случаях, когда все номера $n_1, n_2, \dots, n_N$ различны. Таким образом, в системе одинаковых фермионов в одном и том же квантовом состоянии может находиться не более одной частицы. Иными словами, два и более фермиона не могут находиться в одном и том же квантовом состоянии ({\em принцип запрета Паули}, 1925 г.).
