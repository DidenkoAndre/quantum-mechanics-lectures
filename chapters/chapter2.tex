\chapter{Операторы физических величин}

\section{Условие нормировки волн де Бройля}

\begin{equation}
\label{eq:2_1_1}
\Psi_{\vp}(\vr,t) = A e^{\frac{i}{\hbar}(\vp\vr-Et)}
\end{equation}

Применим условие нормировки \eqref{eq:1_4_3}:

$$\int \Psi^*_{\vsp}(\vr,t) \Psi_{\vp}(\vr,t) dv = |A|^2 e^{\frac{i}{\hbar}(E' - E)t} \int e^{\frac{i}{\hbar}(\vp - \vsp)\vr} \, dv$$

Основные свойства $\delta$-функции:
\begin{enumerate}
\item $ \delta {\vr} = \delta (x) \cdot \delta (y) \cdot \delta (z) $
\item $ \delta (\alpha x) = \frac{1}{|\alpha|} \delta (x), \; (\alpha \ne 0) $
\item $~ f(x) \delta (x - a) = f(a) \delta(x-a)$
\end{enumerate}

Применяя эти свойства, получим:

$$\int e^{\frac{i}{\hbar}(\vp - \vsp)\vr} \, dv \equiv (2 \pi )^3 \delta \left( \frac{\vp - \vsp}{\hbar} \right) = (2 \pi \hbar)^3 \delta (\vp - \vsp)$$

$$\int \Psi^*_{\vsp}(\vr,t) \Psi_{\vp}(\vr,t) dv = |A|^2 e^{\frac{i}{\hbar}(E' - E)t} (2 \pi \hbar)^3 \delta (\vp - \vsp) = |A|^2 (2 \pi \hbar)^3 \delta (\vp - \vsp)$$

Отсюда: $$A = \frac{1}{(2 \pi \hbar)^{3/2}}$$

Из \eqref{eq:2_1_1} получим выражение для волны де Бройля с учётом нормировки:
\begin{equation}
\label{eq:2_1_2}
\boxed{\Psi_{\vp}(\vr,t) = \frac{1}{(2 \pi \hbar)^{3/2}} e^{\frac{i}{\hbar} (\vp \vr - Et)}}
\end{equation}

Условие нормировки волны де Бройля на $\delta$-функцию:
\begin{equation}
\label{eq:2_1_3}
\boxed{\int\Psi^*_{\vsp}(\vr,t) \, \Psi_{\vp}(\vr,t) \, dv = \delta (\vp - \vsp)}
\end{equation}

Волновой пакет:
\begin{equation}
\label{eq:2_1_4}
\Psi(\vr,t) = \int d^3p \; c(\vp) \, \Psi_{\vp}(\vr,t)
\end{equation}
где $c(\vp)$ --- весовая функция.

$$\left. \int\Psi^*_{\vsp}(\vr,t) \, \Psi(\vr,t) \, dv \right|_{(2.1.4)} = \left. \int d^3p \, c(p) \int \, \Psi^*_{\vsp}(\vr,t) \, \Psi_{\vp}(\vr,t) dv  \right|_{(2.1.3)} = $$
$$= \int d^3p \, c(\vp) \cdot \delta (\vp - \vsp) = c(\vsp)$$

Таким образом:
\begin{equation}
\label{eq:2_1_5}
c(\vp) = \int\Psi^*_{\vp}(\vr,t) \, \Psi(\vr,t) \, dv
\end{equation}

\begin{equation}
\label{eq:2_1_6}
\int \abs{ \Psi(\vr,t) }^2 \, dv = \int \abs{c(\vp)}^2 \, d^3 p
\end{equation}

\begin{excr}
Доказать равенство \eqref{eq:2_1_6}
\end{excr}

\begin{equation}
\label{eq:2_1_7}
\int \abs{c(\vp)}^2 \, d^3 p = 1
\end{equation}
Здесь подыинтегральное выражение равно вероятности обнаружить импульс частицы, описываемой волновой функцией $\Psi_{\vp}(\vr,t)$, в интервале $(\vp,\vp+d\vp)$.

$c(\vp)$ --- \underbar{волновая функция в импульсном пространстве}.

\section{Среднее значение координаты и импульса. Операторы координаты и импульса.}

$\avg{F} \rightarrow \op{F}$ --- оператор, действующий в пространстве состояний $\Psi(\vr,t)$.

\begin{defn}
Оператор физической величины:
\begin{equation}
\label{eq:2_2_1}
\op{F} = \avg{\op{F}}_\Psi \equiv \int \Psi^*(\vr,t) \op{F} \Psi(\vr,t) dv,
\end{equation}
где $\avg{\op{F}}_\Psi$ понимается в смысле квантово-механического среднего.
\end{defn}

\begin{equation}
\label{eq:2_2_2}
\begin{gathered}
\avg{F(\vr)} = \int F(\vr) \, dp = 
\left. \int F(\vr) \rho(\vr,t) \, dv  \right|_{\text{\eqref{eq:1_4_1} и \eqref{eq:1_4_2}}} = \\
= \int F(\vr) \abs{\Psi(\vr,t)}^2 \, dv = \int \Psi^* F(\vr) \Psi \, dv
\end{gathered}
\end{equation}

\begin{equation}
\label{eq:2_2_3}
\avg{F(\vp)} = \int F(\vp) \abs{c(\vp)}^2 \, d^3 p = \int c^*(\vp) F(\vp) c(\vp) d^3 p 
\end{equation}

Возьмем $F(\vr) = \vr$ --- координатный радиус-вектор.

$$
\left. \avg{\vr} \right|_{\text{\eqref{eq:2_2_2}}} = 
\left. \int \Psi^*(\vr, t) \, \vr \, \Psi(\vr,t) \, dv \right|_{\text{\eqref{eq:2_2_1}}} = \avgh{\op{\vr}}_\Psi \equiv
\int \Psi^*(\vr,t) \op{\vr} \Psi(\vr,t) \, dv
$$

Следовательно:
\begin{equation}
\label{eq:2_2_4}
\boxed{\op{\vr} = \vr}
\end{equation}

В координатном пространстве (в координатном представлении) оператор координаты совпадает с самим значением координаты частицы.

\begin{equation}
\label{eq:2_2_5}
\op{F}(\vr) = F(\vr) = U(\vr) ~~ \Rightarrow ~~ \op{U}(r) = U(r)
\end{equation}

\begin{equation}
\label{eq:2_2_6}
\left. \avg{\vp} \right|_{\text{\eqref{eq:2_2_3}}} = 
\left. \int d^3 p \, \abs{c(\vp)}^2 \vp \right|_{\text{\eqref{eq:2_2_1}}} = 
\int dv \, \Psi^*(\vr,t) \op{\vp} \Psi(\vr,t)
\end{equation}

$$
\avg{\vp} = \left. \int d^3 p \, c^*(\vp) \, \vp \, c(\vp) \right|_{\text{\eqref{eq:2_1_5}}} =
\int d^3 p \, c^*(\vp) \brc{ \int dv \, \vp \Psi_{\vp}^*(\vr,t) \Psi(\vr,t) }
$$

Из \eqref{eq:2_1_2}:
$$
\vp \Psi_{\vp}^*(\vr,t) = i\hbar \nabla \Psi_{\vp}^*(\vr,t)
$$

$$
\avg{\vp} = \int d^3 p \, c^*(\vp) \brc{i\hbar \int dv \, \Psi(\vr,t) \nabla \Psi_{\vp}^*(\vr,t)}
$$

$$
\int dv \, \Psi(\vr,t) \nabla \Psi_{\vp}^*(\vr,t) = 
\int dv \, \nabla \brc{\Psi(\vr,t) \Psi_{\vp}^*(\vr,t)} - \int dv \, \Psi_{\vp}^*(\vr,t) \nabla \Psi(\vr,t)
$$

$$
\begin{gathered}
\avg{\vp} = \int d^3 p \, c^*(\vp) \brc{\int dv \, \Psi_{\vp}^*(\vr,t) (-i\hbar\nabla) \Psi(\vr,t)} = \\
=\int dv \underbrace{ \int d^3 p  \, c^*(\vp) \Psi_{\vp}^*(\vr,t) }_{\Psi^*(\vr,t) \text{ из \eqref{eq:2_1_4}}} (-i\hbar\nabla) \Psi(\vr,t) = \int dv \, \Psi^*(\vr,t) (-i\hbar\nabla) \Psi(\vr,t)
\end{gathered}
$$

Сравнивая полученное выражение с правой частью \eqref{eq:2_2_6}:
\begin{equation}
\label{eq:2_2_7}
\boxed{ \op{\vp} = -i\hbar\nabla }
\end{equation}

\section{Постановка задачи на собственные функции и собственные значения операторов}

\begin{equation}
\label{eq:2_3_1}
\op{\vp} \Psi_\vp(\vr,t) = \vp \Psi_\vp(\vr,t)
\end{equation}

Пусть $\vec{q}\equiv(q_1,...,q_n)$ означает вектор координатного пространства.

Задача на собственые значения $f$ и отвечающие им собственные функции $\Psi_f(\vec{q})$ оператора $\op{F}$: 
\begin{equation}
\label{eq:2_3_2}
\Psi_f(\vec{q}):~~ \op{F}\Psi_f(\vec{q}) = f \Psi_f(\vec{q}), 
\end{equation}

\begin{defn}
Набор собственных значений $\{f\}$ называется \underbar{спектром} оператора $\op{F}$.
\end{defn}

\underbar{Дискретный спектр}:
$$\{f\} = f_1,f_2,...,f_n,... $$

Введем обозначение: $\psi_{f_n}(\vec{q}) \equiv \psi_n(\vec{q})$.

Задача на собственные пары в дискретном спектре:
\begin{equation}
\label{eq:2_3_3}
\op{F} \psi_n(\vec{q}) = f_n \psi_n(\vec{q})
\end{equation}

\underbar{Непрерывный спектр}: Собственные значения оператора $\op{\vp}$ $p \in (-\infty, +\infty)$
