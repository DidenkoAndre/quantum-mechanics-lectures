\chapter{Операторы физических величин}

\section{Условие нормировки волн де Бройля}

Согласно гипотезе де Бройля, состояние свободной частицы с импульсом $\vp$ описывается плоской волной (волной де Бройля)

\begin{equation}
\label{eq:2_1_1}
\Psi_{\vp}(\vr,t) = A e^{\frac{i}{\hbar}(\vp\vr-Et)}
\end{equation}


Коэффициент $A$ должен быть определён из условия нормировки волновой функции \eqref{eq:1_4_3}. Однако, рассматривая свободную частицу во всём пространстве, выше мы убедились, что нормировочный интеграл расходится. Это одна из трудностей квантовой механики.

Проблему расходимости нормировочного интеграла можно решить следующим образом. Рассмотрим интеграл по бесконечному объему от двух волн де Бройля с импульсами $\vsp$ и $\vp$:

$$
\int \Psi^*_{\vsp}(\vr,t) \Psi_{\vp}(\vr,t)\diff v = 
  \abs{A}^2 e^{\frac{i}{\hbar}(E' - E)t} \int e^{\frac{i}{\hbar}(\vp - \vsp)\vr}\diff v
$$%
%
Интеграл в правой части этого равенства определён в классе обобщённых функций через трёхмерную $\delta$-функцию Дирака\footnotemark{}. Основные свойства $\delta$-функции:
%people
\footnotetext{Поль Дир\'{а}к (Paul Dirac, 1902-1984)}
%
\begin{itemize}
\item $\delta {\vr} = \delta (x) \cdot \delta (y) \cdot \delta (z) $
\item $\delta (\alpha x) = \frac{1}{|\alpha|} \delta (x)$, где $\alpha \ne 0$ -- произвольная постоянная, см.~\llref{5}{3}.
\item $f(x) \delta (x - a) = f(a) \delta(x-a)$
\end{itemize}%
%
Применяя первые два свойства, получим:
$$
\int e^{\frac{i}{\hbar}(\vp - \vsp)\vr}\diff v \equiv
  (2 \pi )^3 \delta \brc{\frac{\vp - \vsp}{\hbar}} =
  (2 \pi \hbar)^3 \delta (\vp - \vsp)
$$%
%
Откуда, с помощью третьего свойства, следует
$$
\int \Psi^*_{\vsp}(\vr,t) \Psi_{\vp}(\vr,t)\diff v = 
  \abs{A}^2 e^{\frac{i}{\hbar}(E' - E)t} (2 \pi \hbar)^3 \delta (\vp - \vsp) =
  \abs{A}^2 (2 \pi \hbar)^3 \delta (\vp - \vsp)
$$%
%
Выбирая нормировочный множитель $A = 1/(2 \pi \hbar)^{3/2}$, получим для волны де Бройля \eqref{eq:2_1_1}%
%
\begin{equation}
\label{eq:2_1_2}
\boxed{\Psi_{\vp}(\vr,t) = \frac{1}{(2 \pi \hbar)^{3/2}} e^{\frac{i}{\hbar} (\vp \vr - Et)}}
\end{equation}%
%
а для нормировочного соотношения
\begin{equation}
\label{eq:2_1_3}
\boxed{\int\Psi^*_{\vsp}(\vr,t) \, \Psi_{\vp}(\vr,t)\diff v = \delta (\vp - \vsp)}
\end{equation}%
%
Соотношение \eqref{eq:2_1_3} называется условием нормировки волн де Бройля на $\delta$-функцию.

Если $\vp=\vsp$, то правая часть \eqref{eq:2_1_3} обращается в бесконечность. В этом смысле, как уже ранее было указано, интеграл от квадрата модуля волны де Бройля расходится. Волна де Бройля $\Psi_{\vp}(\vr, t)$ -- это волновая функция состояния, которое не может быть осуществлено (состояние свободного движения частицы со строго определённым импульсом $\vp$). Поэтому нет причин беспокоиться, что волна де Бройля не может быть нормирована на единицу.

Однако, из волн де Бройля можно построить волновой пакет

\begin{equation}
\label{eq:2_1_4}
\Psi(\vr,t) = \int \diff^3 p \; C(\vp) \, \Psi_{\vp}(\vr,t)
\end{equation}%
%
с весовой функцией $C(\vp)$. Зная $\Psi(\vr,t)$, можно найти соответствующую ей функцию $C(\vp)$, т.к. $C(\vp)$ -- это, по существу, Фурье-образ волновой функции $\Psi(\vr,t)$. Действительно:
%

$$
\begin{gathered}
\left. \int \Psi^*_{\vsp}(\vr,t) \, \Psi(\vr,t)\diff v \right|_{\text{\eqref{eq:2_1_4}}} =
  \left. \int \diff ^3p \, C(p) \int \Psi^*_{\vsp}(\vr,t) \, \Psi_{\vp}(\vr,t)\diff v \right|_{\text{\eqref{eq:2_1_3}}} =\\
  = \int \diff^3p \, C(\vp) \cdot \delta (\vp - \vsp) = C(\vsp)
\end{gathered}
$$%
%
Таким образом,
\begin{equation}
\label{eq:2_1_5}
C(\vp) = \int\Psi^*_{\vp}(\vr,t) \, \Psi(\vr,t)\diff v
\end{equation}%
%
Для преобразования Фурье справедливо равенство
\begin{equation}
\label{eq:2_1_6}
\int \abs{ \Psi(\vr,t) }^2\diff v = \int \abs{c(\vp)}^2 \diff ^3 p
\end{equation}

\begin{excr}
Доказать \eqref{eq:2_1_6}.
\end{excr}%

\noindent
Поэтому, если волновой пакет \eqref{eq:2_1_4} нормирован на единицу в координатном пространстве соотношением \eqref{eq:1_4_3}, то и для весовой функции $C(\vp)$ выполняется условие нормировки

\begin{equation}
\label{eq:2_1_7}
\int \abs{C(\vp)}^2\diff^3 p = 1
\end{equation}%
%
в импульсном пространстве. При нормировке \eqref{eq:2_1_7} подынтегральное выражение можно интерпретировать как вероятность найти импульс частицы в интервале $(\vp,\vp+\diff\vp)$. Тогда $C(\vp)$ -- это {ВФ в импульсном пространстве}, т.е. амплитуда вероятности того, что частица, волновая функция которой задана волновым пакетом, обладает импульсом $\vp$.

\section{Среднее значение координаты и импульса. Операторы координаты и импульса.}

Поведение частиц в микромире описывается волновой функцией $\Psi(\vr,t)$, которая носит вероятностный характер, причём даже в том случае, когда описываемая ею система состоит всего лишь из одной-единственной частицы. В связи с этим квантовая механика позволяет определить лишь средние значения физических величин независимо от того, имеется много или одна микрочастица. В квантовой механике каждой физической ({\em наблюдаемой}) величине $F$ ставится в соответствие оператор $\op{F}$, действующий в пространстве волновых функций $\Psi(\vr,t)$, описывающих состояния физической системы. При этом оператор $\op{F}$ определяется через среднее значение соответствующей ему величины $\avg{F}$ в состоянии с волновой функцией $\Psi(\vr,t)$ следующим образом:%
%
\begin{equation}
\label{eq:2_2_1}
\avg{F} = \avg{\op{F}}_\Psi \equiv \int \Psi^*(\vr,t) \, \op{F} \, \Psi(\vr,t)\diff v ~\text{\footnotemark{}}
\end{equation}%
%
\footnotetext{Здесь, согласно общему правилу, оператор действует на функцию, стоящую справа от него.}%
%
где $\avg{\op{F}}_\Psi$ понимается в смысле квантово-механического среднего, при условии, что $\int \Psi^*(\vr,t) \, \Psi(\vr,t)\diff v = 1$. При таком определении оператора его квантовое среднее значение совпадает с наблюдаемым значением $\avg{F}$ физической величины $F$ в состоянии $\Psi(\vr,t)$.

С точки зрения теории вероятности, среднее значение для некоторой функции $F(\vr)$ координаты частицы есть её математическое ожидание

\begin{equation}
\label{eq:2_2_2}
\begin{gathered}
\avg{F(\vr)} = \int F(\vr)\diff p = 
\left. \int F(\vr) \underbrace{\rho(\vr,t)\diff v}_{\diff P}  \right|_{\text{\eqref{eq:1_4_1} и \eqref{eq:1_4_2}}} = \\
= \int F(\vr) \abs{\Psi(\vr,t)}^2\diff v = \int \Psi^*(\vr,t) \, F(\vr) \, \Psi(\vr,t)\diff v
\end{gathered}
\end{equation}%
%
Аналогично, для среднего значения некоторой функции $F(\vp)$ (с учётом вероятностной интерпретации для функции $C(\vp)$ предыдущего параграфа) получим

\begin{equation}
\label{eq:2_2_3}
\avg{F(\vp)} = \int F(\vp) \abs{c(\vp)}^2 \, \diff^3 p = \int c^*(\vp) \, F(\vp) \, c(\vp) \diff^3 p 
\end{equation}

Примем $F(\vr) = \vr$ -- координатный радиус-вектор. Тогда среднее значение координаты частицы есть
$$
\left. \avg{\vr} \right|_{\text{\eqref{eq:2_2_2}}} = 
  \left. \int \Psi^*(\vr, t) \, \vr \, \Psi(\vr,t)\diff v \right|_{\text{\eqref{eq:2_2_1}}} =
  \avgh{\op{\vr}}_\Psi \equiv \int \Psi^*(\vr,t) \, \op{\vr} \, \Psi(\vr,t)\diff v
$$%
%
Следовательно:
\begin{equation}
\label{eq:2_2_4}
\boxed{\op{\vr} = \vr}
\end{equation}%
%
В координатном пространстве ({\em координатном представлении}) оператор координаты совпадает с самим значением координаты частицы. Аналогичное равенство имеет место для произвольных функций координат частицы и её оператора, например для оператора потенциальной энергии в гамильтониане:

\begin{equation}
\label{eq:2_2_5}
\op{U}(\vr) = U(\vr)
\end{equation}

Из общего определения оператора \eqref{eq:2_2_1} для оператора импульса частицы $\vp$ и его среднего значения $\avg{\vp}$ с помощью \eqref{eq:2_2_3} можно написать равенства

\begin{equation}
\label{eq:2_2_6}
\left. \avg{\vp} \right|_{\text{\eqref{eq:2_2_3}}} = 
  \left. \int \diff^3 p \, \abs{C(\vp)}^2 \vp \right|_{\text{\eqref{eq:2_2_1}}} = 
  \int\diff v \, \Psi^*(\vr,t) \op{\vp} \Psi(\vr,t)
\end{equation}%
%
Преобразуем левую часть \eqref{eq:2_2_6}:

$$
\avg{\vp} = \left. \int \diff^3 p \, C^*(\vp) \, \vp \, C(\vp) \right|_{\text{\eqref{eq:2_1_5}}} =
  \int \diff^3 p \, C^*(\vp) \brc{ \int \diff v \, \vp \Psi_{\vp}^*(\vr,t) \, \Psi(\vr,t) }
$$%
%
Для волны де Бройля \eqref{eq:2_1_2} справедливо соотношение

$$
\vp \Psi_{\vp}^*(\vr,t) = i\hbar \nabla \Psi_{\vp}^*(\vr,t)
$$%
%
с учётом которого
$$
\avg{\vp} = \int \diff^3 p \, C^*(\vp) \brc{i\hbar \int \diff v \, \Psi(\vr,t) \nabla \Psi_{\vp}^*(\vr,t)}
$$%
%
Возьмём интеграл в круглых скобках по частям:
$$
\int \diff v \, \Psi(\vr,t) \nabla \Psi_{\vp}^*(\vr,t) = 
\int \diff v \, \nabla \brc{\Psi(\vr,t) \Psi_{\vp}^*(\vr,t)} - \int \diff v \, \Psi_{\vp}^*(\vr,t) \nabla \Psi(\vr,t)
$$%
%
Используем $\left. \Psi(\vr,t) \right|_{r\to\infty}\to0$ в силу квадратичной интегрируемости $\Psi(\vr,t)$ (см.~\eqref{eq:1_4_3})
$$
\begin{gathered}
\avg{\vp} = \int \diff^3 p \, C^*(\vp) \brc{\int \diff v \, \Psi_{\vp}^*(\vr,t) (-i\hbar\nabla) \Psi(\vr,t)} = \\
  =\int \diff v \underbrace{
      \int \diff^3 p  \, C^*(\vp) \Psi_{\vp}^*(\vr,t)
    }_{\Psi^*(\vr,t) \text{ из \eqref{eq:2_1_4}}} (-i\hbar\nabla) \Psi(\vr,t) =
  \int \diff v \, \Psi^*(\vr,t) (-i\hbar\nabla) \Psi(\vr,t)
\end{gathered}
$$%
%
Сравнивая полученное выражение с правой частью \eqref{eq:2_2_6}, получим вид оператора импульса частицы $\vp$ в координатном представлении:
\begin{equation}
\label{eq:2_2_7}
\boxed{ \op{\vp} = -i\hbar\nabla }
\end{equation}

При переходе от классической к квантовой механике мы теряем однозначность определения $\vr$ и $\vp$, но приобретаем взаимосвязи между этими физическими величинами. В классической физике траектория и импульс точно определены и не связаны друг с другом (являются независимыми переменными). В квантовой механике как траектория, так и импульс точно не определены, но их распределения -- амплитуды вероятностей $\Psi(\vr,t)$ и $C(\vp)$ -- связаны друг с другом.

\section{Постановка задачи на собственные функции и собственные значения операторов}

Для волны де Бройля справедливо соотношение
\begin{equation}
\label{eq:2_3_1}
\op{\vp} \Psi_\vp(\vr,t) = \vp \Psi_\vp(\vr,t)
\end{equation}%
%
которое является частным случаем задачи на собственные значения и собственные функции (пары) в квантовой механике.

Пусть $\vec{q}\equiv(q_1,...,q_n)$ -- конфигурационное пространство (пространство обобщённых координат физической системы), $\vec{q}$ -- действительный вектор. Задача о поиске функций $\psi_f(\vec{q})$ удовлетворяющих уравнению обобщённого вида

\begin{equation}
\label{eq:2_3_2}
\op{F}\psi_f(\vec{q}) = f \psi_f(\vec{q}), 
\end{equation}%
%
называется задачей о нахождении собственных значений $f$ оператора $\op{F}$ и отвечающих им собственных функций $\psi_f(\vec{q})$. Набор всех значений $\brcr{f}$ называется {\em спектром оператора} $\op{F}$.

Если спектр \underline{дискретен}, т.е. $\{f\} = f_1,f_2,...,f_n,... $, то пользуются обозначением $\psi_{f_n}(\vec{q}) \equiv \psi_n(\vec{q})$. Тогда%
%
\begin{equation}
\label{eq:2_3_3}
\op{F} \psi_n(\vec{q}) = f_n \psi_n(\vec{q})
\end{equation}%
%
Однако спектр может быть и \underline{непрерывным}. В частности, в одномерном случае \eqref{eq:2_3_1} спектром оператора импульса $\vp$ является вся действительная ось: $p \in (-\infty, +\infty)$.
