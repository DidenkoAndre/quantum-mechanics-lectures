\chapter{Атом водорода}

\section{Атом водорода. Атомная система единиц}

Рассмотрим одну из важнейших задач, связанных с движением в центрально-симметричном поле, а именно — движением электрона с зарядом $-e$ и массой $m$ в кулоновском поле ядра с зарядом $+Ze$:
$$
U(r) = -\frac{Ze^2}{r}
$$%
%
где $Z>1$ для водородоподобного иона (атома), $Z=1$ для атома водорода. Заметим, что для атома водорода квантово-механическая задача двух взаимодействующих тел сводится к задаче о движении одной частицы с приведённой массой $\mu = (m M)/(m + M)$ (упр. 11 2-го задания). Но здесь мы будем считать ядро бесконечно тяжёлым, т.е. $M \to \infty$, так что $\mu = m$.

Радиальное уравнение \eqref{eq:9_2_1} приобретает вид
\begin{equation}
\label{eq:10_1_1}
\brs{\D{^2}{r^2} + \frac{2}{r}\D{}{r}} R(r) + 
\brcr{-\frac{l(l+1)}{r^2} + \frac{2m}{\hbar^2}\frac{Ze^2}{r} + \frac{2mE}{\hbar^2}} R(r) = 0
\end{equation}%
%
здесь опущены индексы $n$ и $l$ при радиальной части волновой функции. Решения уравнения \eqref{eq:10_1_1} удобно получить в атомной системе единиц: $\hbar = m = e = 1$. При этом атомная единица (а.е) длины или {\em боровский радиус}:
$$
a = \frac{\hbar^2}{me^2} = 1\text{~а.е.} = 0.529 \cdot 10^{-8} \text{~см} = 0.529 \text{~\AA}
$$%
%
атомная единица энергии:
$$
E_a = \frac{e^2}{a} = \frac{me^4}{\hbar^2} = 1 \text{~а.е.} = 27.21 \text{~эВ}
$$%
%
В безразмерных переменных $\rho = r/a$ и $\epsilon = E/E_a$ получаем уравнение \eqref{eq:10_1_1} в виде
\begin{equation}
\label{eq:10_1_2}
\boxed{
	\brcr{\D{^2}{\rho^2} + \frac{2}{\rho} \D{}{\rho}} R(\rho) + \brcr{-\frac{l(l+1)}{\rho^2} + \frac{2Z}{\rho} - \kappa^2} R(\rho) = 0
}
\end{equation}%
%
где $-\kappa^2 = 2 \epsilon = 2 \frac{E}{E_a} < 0$, т.е. рассматриваются связанные состояния дискретного спектра.

\section{Энергетический спектр и радиальные волновые функции стационарных состояний атома водорода. Главное и радиальное квантовые числа}

В силу условия нормировки \eqref{eq:9_2_2} не должно быть неограниченных решений \eqref{eq:10_1_2}, т.е. на краях области определения $\rho \to 0$ и $\rho \to \infty$ необходимо наложить граничные условия:
\begin{enumerate}
\item $\left. R(\rho)\right|_{\rho \to 0} \to \const$
\item $\left. R(\rho)\right|_{\rho \to \infty} \to 0$
\end{enumerate}


В пределе $\rho \to 0$ уравнение \eqref{eq:10_1_2} принимает форму:
$$
\brc{\D{^2}{\rho^2} + \frac{2}{\rho}\D{}{\rho} - \frac{l(l+1)}{\rho^2}} R_0(\rho) = 0
$$

Ищем решение этого уравнения в виде степенной функции $R_0(\rho) \sim \rho^q$, тогда для показателя $q$ получается:
$$
\underbrace{q (q - 1) + 2q}_{q(q+1)} - l(l + 1) = 0
$$
т.е. возможны два решения: $q_1 = l$, $q_2 = -(l+1)$. Очевидно, второе решение не удовлетворяет поставленному граничному условию: оно обращается в бесконечность при $\rho \to 0$ (напомним, что $l \geqslant 0$). Поэтому:
$$
\boxed{
	\left. R(\rho)\right|_{\rho \to 0} \sim \rho^l
}
$$

Найдём теперь асимптотику решения \eqref{eq:10_1_2} при $\rho \to \infty$:
$$
\D{^2}{\rho^2}R_\infty(\rho) - \kappa^2 R_\infty(\rho) = 0
$$

Отсюда $\boxed{\left. R(\rho)\right|_{\rho \to \infty} \sim e^{-\kappa \rho}}$, т.к. другое решение ($\sim e^{\kappa \rho}$) при $\rho \to \infty$ неограниченно возрастает. Очевидно, что решение уравнения \eqref{eq:10_1_2} во всей области определения $\rho$ следует искать в виде:
\begin{equation}
\label{eq:10_2_1}
\boxed{
	R(\rho) = e^{-\kappa \rho} \rho^l v(\rho)
},
\end{equation}

где для искомой функции $v(\rho)$ есть ограничения на её экспоненциальный рост на бесконечности. Имеем:
$$
\begin{gathered}
R'(\rho) = e^{-\kappa \rho} \rho^l \brs{v' + \brc{\frac{l}{\rho} - \kappa} v}\\
R''(\rho) = e^{-\kappa \rho} \rho^l \brs{v'' + \brc{\frac{l}{\rho} - \kappa}v' - \frac{l}{\rho^2}v + \brc{\frac{l}{\rho} - \kappa} \brs{v' + \brc{\frac{l}{p} - \kappa}v} } = \\ = e^{-\kappa \rho} \rho^l \brs{v'' + 2 \brc{\frac{l}{\rho} - \kappa} v' + \brc{ \frac{l(l-1)}{\rho^2} - \frac{2\kappa l}{\rho} + \kappa^2 } v }
\end{gathered}
$$

Из \eqref{eq:10_1_2} получаем:
$$
v'' + 2\brc{\frac{l}{\rho} - \kappa}v' + \brc{ \frac{l(l-1)}{\rho^2} - \frac{2\kappa l}{\rho} + \kappa^2 } v + \frac{2}{\rho}v' + \frac{2}{\rho} \brc{ \frac{l}{\rho} - \kappa} v + 
$$
$$
+\brc{- \frac{l(l+1)}{\rho^2} + \frac{2Z}{\rho} - \kappa^2}v = 0
$$

В итоге:
\begin{equation}
\label{eq:10_2_2}
\boxed{
	\rho v'' + v' \brc{2(l+1) - 2\kappa \rho} + v\brc{2Z - 2\kappa(l+1)} = 0
}
\end{equation}

Решение \eqref{eq:10_2_2} будем искать в виде степенного ряда:
\begin{equation}
\label{eq:10_2_3}
v(\rho) = \sum_{k=0}^{\infty} a_k \rho^k
\end{equation}

Отсюда:
$$
\begin{gathered}
v'(\rho) = \sum_{k=0}^{\infty} a_k k \rho^{k-1} = \sum_{k=0}^{\infty} a_{k+1} (k+1) \rho^k \\
v''(\rho) = \sum_{k=0}^{\infty} a_k k (k-1) \rho^{k-2} = \sum_{k=0}^{\infty} a_{k+1} (k+1) k \rho^{k-1}
\end{gathered}
$$

Подставляя в \eqref{eq:10_2_2}, получаем:
$$
\sum_{k=0}^{\infty} \rho^k \brs{a_{k+1} \brc{k(k+1) + 2(l+1)(k+1)} + a_k\brc{2Z - 2\kappa(l+1) - 2\kappa k} } = 0
$$

откуда следует рекуррентное соотношение для коэффициентов ряда:
\begin{equation}
\label{eq:10_2_4}
\boxed{
	a_{k+1} = a_k \frac{2\brc{\kappa(l+1+k)-Z}}{(k+1)\brc{k+2(l+1)}}
}
\end{equation}

Из соотношения \eqref{eq:10_2_4} видно, что при $k \gg 1$ все слагаемые будут одного знака. Значит, при $\rho \to \infty$ основной вклад в $v(\rho)$ будут давать слагаемые с большими $k$. При $k \gg 1$:
$$
{ \abs{\frac{a_{k+1}}{a_k}} }_{k \gg 1} \sim \frac{2\kappa k}{k^2} = \frac{2\kappa}{k}
$$
однако:
$$
e^{2\kappa \rho} = 1 + \frac{2\kappa \rho}{1!} + ... + \frac{(2\kappa \rho)^k}{k!} + ...
$$
и для ряда растущей экспоненты:
$$
{ \abs{\frac{a_{k+1}}{a_k}} }_{k \gg 1} \sim \frac{(2\kappa)^{k+1} k!}{(k+1)! (2\kappa)^k} \sim \frac{2\kappa}{k}
$$

Таким образом, полученный ряд \eqref{eq:10_2_3} для $v(\rho)$ асимптотически ведёт себя как
$$
\left. v(\rho) \right|_{\rho \to \infty} \sim e^{2\kappa \rho}
$$

При такой асимптотике, согласно \eqref{eq:10_2_1}, радиальная волновая функция расходится на бесконечности, т.е.:
$$
\left. R(\rho) \right|_{\rho \to \infty} \sim e^{\kappa \rho}
$$

Поэтому суммирование в \eqref{eq:10_2_3} может происходить только в конечных пределах, иными словами, ряд \eqref{eq:10_2_3} должен <<обрываться>> и переходить в конечный полином некоторой степени $k=n_r$, т.е. $a_{n_r} \neq 0$, но при любом $k > n_r$ $a_k \equiv 0$.

Из \eqref{eq:10_2_4} следует условие <<обрыва>> ряда \eqref{eq:10_2_3}:
\begin{equation}
\label{eq:10_2_5}
\boxed{
	\frac{Z}{\kappa} - l - 1 = n_r = 0, 1, 2, ...
}
\end{equation}

Число $n_r \geqslant 0$ определяет степень полинома \eqref{eq:10_2_3} и соответственно число его нулей (или число узлов радиальной волновой функции $R(\rho)$ в \eqref{eq:10_2_1}, не считая точки $\rho = 0$). Число $n_r$ называют {\em радиальным квантовым числом}. Здесь мы имеем частный случай осцилляционной теоремы одномерного движения (см. конец \S 3 главы VII).

Следуя \eqref{eq:10_2_5}, положим по определению:
\begin{equation}
\label{eq:10_2_6}
\boxed{
	n = n_r + l + 1 = 1, 2, ...
}
\end{equation}

натуральное число $n \in \mathbb{N}$, которое называют {\em главным квантовым числом}. При этом $\boxed{n_r = n - l - 1 \geqslant 0}$ и получается ограничение на возможные значения орбитального момента: $\boxed{0 \leqslant l \leqslant n - 1}$. Тогда из \eqref{eq:10_1_2}, \eqref{eq:10_2_5} и \eqref{eq:10_2_6} получается энергетический спектр водородоподобного атома:
\begin{equation}
\label{eq:10_2_7}
\begin{gathered}
\epsilon_n  = \frac{E_n}{E_a} = - \frac{\kappa^2}{2} = \boxed{- \frac{Z^2}{2 n^2} = \epsilon_n } \\
\boxed{E_n = - \frac{Z^2 me^4}{2\hbar^2} \frac{1}{n^2}}
\end{gathered}
\end{equation}

Соответственно \eqref{eq:10_2_1} радиальная волновая функция имеет вид
$$
R_{nl}(\rho) = R_{n_r l}(\rho) = C_{nl} e^{-\kappa \rho}\rho^l v_{n_r l}(\rho)
$$
где определяемые рекуррентными соотношениями \eqref{eq:10_2_4} полиномы называют {\em обобщёнными (присоединёнными) полиномами Лагерра}\footnotemark{}
%people
\footnotetext{Эдмон Никола Лагерр (Edmond Nicolas Laguerre, 1834-1886)}
%
$$
\begin{gathered}
\boxed{v_{n_r l}(\rho) = L_{n_r}^{2l+1} (2 \kappa \rho) }, \text{~где} \\
L_{s}^{k}(x) = e^x x^{-s} \D{^k}{x^k}(e^{-x} x^{s+k})
\end{gathered}
$$
Коэффициент $C_{nl}$ определяется из условия нормировки \eqref{eq:9_2_2} для радиальной волновой функции:
$$
\int_{0}^{\infty} \abs{R_{nl}(\rho)}^2 \rho^2 \diff\rho = 1
$$
Состояние атома водорода определяется волновой функцией
$$
\Psi_{nlm}(\vr) = R_{nl}(r)Y_{lm}(\theta, \phi)
$$
которая, например, для основного $1s$-состояния имеет вид
$$
\boxed{\Psi_{100}(\vr)} = \underbrace{R_{10}(r)}_{\frac{2}{\sqrt{a^3}} e^{-r/a}} \underbrace{Y_{00}(\theta, \phi)}_ {\frac{1}{\sqrt{4\pi}}} \boxed{= \frac{1}{\sqrt{\pi a^3}} e^{-r/a}}
$$

\begin{sloppypar}
\section{Кратность вырождения уровней. Кулоновское (случайное) вырождение}
\end{sloppypar}

Из \eqref{eq:10_2_7} видно, что спектр водородоподобного атома является вырожденным. Энергия определяется только главным квантовым числом, а кроме него есть ещё 2 квантовых числа:
\begin{itemize}
\item $l = 0, 1, ..., n-1$ -- орбитальное
\item $m = 0, \pm 1, ..., \pm l$ -- магнитное
\end{itemize}

Уровни энергии не зависят от этих квантовых чисел! Кратность вырождения $n$–го уровня равна
$$
g(n) = \sum_{l=0}^{n-1} \sum_{m=-l}^{l} 1 = \sum_{l=0}^{n-1} (2l + 1) = n \frac{1+(2n-1)}{2} = n^2
$$
\noindent (из суммы арифметической прогрессии).

Дальнейший анализ удобно провести, если воспользоваться утверждением следующей теоремы.

\begin{thm}
\label{thm:1_1}
Если $\brs{\op{A}, \op{H}} = 0$, $\brs{\op{B}, \op{H}} = 0$, но $\brs{\op{A}, \op{B}} \neq 0$, то спектр оператора $\op{H}$ вырожден.
\end{thm}

\begin{proof}
\begin{enumerate}
\item Предположим, что оператор $\op{H}$ имеет хотя бы один собственный вектор:
$$
\op{H}\ket{\psi_i} = E_i \ket{\psi_i}
$$%
%
Тогда
$$
\op{A}\op{H}\ket{\psi_i} = \op{H}\op{A}\ket{\psi_i} = E_i \op{A}\ket{\psi_i}
$$%
%
т.е. $\op{A}\ket{\psi_i}$ -- тоже собственный вектор оператора $\op{H}$, отвечающий тому же самому собственному значению $E_i$. Если спектр $\op{H}$ невырожден, то каждому $E_i$ должен отвечать в точности один собственный вектор, $\op{A}\ket{\psi_i} = A_i \ket{\psi_i}$ (т.е. $\ket{\psi_i}$ является также собственным вектором и $\op{A}$).

\item Аналогичное рассмотрение для $\op{B}$ даёт:
$$
\op{B}\ket{\psi_i} = B_i \ket{\psi_i}
$$%
%
Отсюда будет следовать
$$
(\op{A}\op{B} - \op{B}\op{A}) \ket{\psi_i} = (A_i B_i - B_i A_i) \ket{\psi_i} = 0
$$%
%
если собственные значения оператора $\op{H}$ невырожденные. Это верно при любых $E_i$ из множества собственных значений. Но такое равенство, справедливое для всех собственных векторов, образующих полную систему, означало бы $\brs{\op{A}, \op{B}} = 0$, что противоречит утверждению теоремы. Значит, спектр оператора $\op{H}$ вырожден.
\end{enumerate}
\end{proof}

В рамках этой теоремы вырождение по магнитному квантовому числу $m$ характерно для любого центрально-симметричного поля, т.к.
$$
\brs{\op{H}, \op{l}_\alpha} = 0 ~~\to~~ 
\left\{
  \begin{array}{l}
    {[} \op{H}, \op{l}_z {]} = 0 \\
    {[} \op{H}, \op{l}_{\pm} {]} = 0
  \end{array}
\right. ~~\text{\textbf{но:}}~ \brs{\op{l}_z, \op{l}_\pm} \neq 0
$$

Поэтому все $2l+1$ состояний, где $m =-l, (-l+1), (-l+2) ... , l$, отвечают одному уровню энергии.

Однако, помимо вырождения по магнитному квантовому числу $m$, обязательному для любого сферически симметричного поля, в кулоновом поле для всех уровней имеет место {\em дополнительное вырождение} по орбитальному квантовому числу $l$, которое называют ещё <<случайным>>. Природа кулоновского вырождения связана с высокой симметрией кулоновского поля и наличием ещё одного интеграла движения – {\em вектора Рунге--Ленца}\footnotemark{}:
%people
\footnotetext{Карл Рунге (Carl Runge, 1856-1927); Вильгельм Ленц (Wilhelm Lenz, 1888-1957)}
%
$$
\op{\vec{A}} = \frac{\op{\vr}}{r} + \frac{1}{2Z} \brc{\op{\vec{l}} \times \op{\vp} - \op{\vp} \times \op{\vec{l}}} ~~~ \text{(в а.е.)}
$$

коммутирующего с $\op{H} = \op{\vp^2}/2 - Z/r$ (в а.е), но не коммутирующего с $\op{\vec{l}^2}$. Тогда, в соответствии с теоремой, если $\brs{\op{H}, \op{A}_\alpha} = 0$ (доказательство приведено в \autoref{runge}), $\brs{\op{H}, \op{\vec{l}^2}} = 0$, но $\brs{\op{\vec{l}^2}, \op{A}_\alpha} \neq 0$, то это автоматически ведёт к вырождению собственных значений $\op{H}$ по $l$.

\begin{excr}
Доказать, что $\brs{\op{\vec{l}^2}, \op{A}_\alpha} \neq 0$
\end{excr}
