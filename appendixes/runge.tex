\chapter{Сохранение векторного оператора Рунге-Ленца}
\label{runge}

\renewcommand{\thesection}{}
\renewcommand{\theequation}{\arabic{equation}}

\hfill \begin{minipage}[h]{0.65\textwidth}
\textit{
Говорит как-то Лифшиц Ландау:\\ 
--- Что делать? Я целую тетрадь выкладок потерял в трамвае!\\
--- Да ничего страшного, напишем как всегда: <<откуда очевидно...>>}
\begin{flushright}
Народный юмор
\end{flushright}
\end{minipage}
\vspace{0.5cm}

В квантовой механике оператор интеграла движения коммутирует с независящим явно от времени гамильтонианом. В кулоновском поле имеет место сохранение векторного оператора Рунге-Ленца $\op{\vec{A}}$, т.е. $[\op{H},\op{\vec{A}}] = 0$. \\
Учебные пособия, в частности <<Квантовая механика>> Ландау и Лифшица, приводят данный факт без доказательства, с указанием, что это <<легко проверяется>>\footnote{Ландау Л.Д., Лифшиц Е.М. Квантовая механика (нерелятивистская теория). — Физматлит, 2002. — 808 с. — («Теоретическая физика», том III), \S 36, пункт <<Природа кулонова вырождения>>}, чему посвящено это приложение.

\begin{stmt*}
\begin{equation}
\label{eq:1}
\op{H}=\left. \frac{\op{\vp}^2}{2}-\frac{Z}{r} \right|_{Z=1} \equiv \frac{1}{2} \brc{\op{\vp}^2 - \frac{2}{r}}
\end{equation}

\begin{equation}
\label{eq:2}
\op{\vec{A}} = \frac{\op{\vr}}{r} + \frac{1}{2} \brc{\op{\vec{l}} \times \op{\vp}} - \frac{1}{2} \brc{ \op{\vp} \times \op{\vec{l}}}
\end{equation}
$$
\begin{array}{lll}
\op{\vr} = r_i  ~~~& \displaystyle \op{\vp} = -\mathbf{i} \pd{}{r_i}  ~~~&  \vec{l} = \op{\vr} \times \op{\vp}
\end{array}
$$
Доказать, что $[\op{H},\op{\vec{A}}] = 0$
\end{stmt*}

\begin{proof}
Из \cref{ex:1_1_6}:
\begin{equation}
\label{eq:eq1}
\begin{gathered}
{[\op{A} \op{B}, \op{C}]} = \op{A} [\op{B},\op{C}] + [\op{A},\op{C}]\op{B} \\
{[\op{A}, \op{B} \op{C}]} = \op{B}[\op{A},\op{C}] + [\op{A},\op{B}]\op{C}
\end{gathered}
\end{equation}

Вычислим некоторые коммутаторы и произведения, которые будут необходимы при доказательстве главной задачи

\begin{equation}
\label{eq:eq5}
\begin{gathered}
\brs{\op{p_i}, \frac{1}{r}} \bullet = \brs{\op{p_i}, \frac{1}{\sqrt{r_m r^m}}} \bullet = \\ = 
-\mathbf{i} \pd{}{r_i} \brc{\frac{1}{\sqrt{r_m r^m}} \bullet} + \frac{1}{\sqrt{r_m r^m}} \mathbf{i} \pd{\bullet}{r_i} = \mathbf{i} \frac{\op{r}_i}{(\sqrt{r_m r^m})^3} \bullet = \mathbf{i} \frac{\op{r}_i}{r^3} \bullet
\end{gathered}
\end{equation}\\

\begin{equation}
\label{eq:eq2}
\begin{gathered}
\brs{\frac{1}{r}, \op{\vec{l}}} = \brs{\frac{1}{r}, \op{\vr} \times \op{\vp}} = \epsilon_{ijk} \brs{ \frac{1}{r}, \op{r_j}\op{p_k} } = \\ =
\left. \epsilon_{ijk} \brc{ \op{r_j} \brs{ \frac{1}{r}, \op{p_k} } - \underbrace{ \brs{ \frac{1}{r}, \op{r_j} } }_{=0} \op{p_k} } \right|_{\text{\eqref{eq:eq5}}} = 
-\mathbf{i} \epsilon_{ijk} \op{r}_j \frac{\op{r}_k}{r^3} = 0
\end{gathered}
\end{equation}\\

\begin{equation}
\label{eq:eq3}
\begin{gathered}
\brs{\frac{1}{r^3},\op{p}_k} \bullet = \frac{1}{(\sqrt{r_m r^m})^3} \mathbf{i} \pd{\bullet}{r_k}  + \mathbf{i} \pd{}{r_k} \brc{\frac{1}{(\sqrt{r_m r^m})^3} \bullet} = \\ = -\mathbf{i} \frac{3 \cdot 2 \op{r}_k}{2 (\sqrt{r_m r^m})^5} \bullet = -\mathbf{i} \frac{3 \op{r}_k}{r^5} \bullet
\end{gathered}
\end{equation}\\

\begin{equation}
\label{eq:eq4}
\begin{gathered}
\brs{\frac{1}{r^3}, \op{\vec{l}}} = \brs{\frac{1}{r^3}, \op{\vr} \times \op{\vp}} = \epsilon_{ijk} \brs{ \frac{1}{r^3}, \op{r_j}\op{p_k} } = \\ = 
\left. \epsilon_{ijk} \brc{ \op{r_j} \brs{ \frac{1}{r^3}, \op{p_k} } - \underbrace { \brs{ \frac{1}{r^3}, \op{r_j} } }_{=0} \op{p_k} } \right|_{\text{\eqref{eq:eq3}}} =
-3 \mathbf{i} \epsilon_{ijk} \op{r}_j \frac{\op{r}_k}{r^5} = 0
\end{gathered}
\end{equation}\\

\begin{equation}
\label{eq:eq6}
\brs{\op{p}_i, \frac{\op{r}_j}{r}} = \op{r}_j \brs{\op{p}_i, \frac{1}{r}} + \brs{\op{p}_i, \op{r}_j} \frac{1}{r} = \mathbf{i} \brc{\frac{\op{r}_i\op{r}_j}{r^3}-\frac{\delta_{ij}}{r}}
\end{equation}\\

\begin{equation}
\label{eq:eq7}
(\op{\vr} \times \op{\vec{l}})_i = \epsilon_{ijk}\epsilon_{abk} \op{r}_j \op{r}_a \op{p}_b = \op{r}_j \op{r}_i \op{p}_k - \op{r}_j \op{r}_j \op{p}_i = \op{r}_i (\op{\vr} \cdot \op{\vp}) - r^2 \op{p}_i
\end{equation}

\begin{equation}
\label{eq:eq8}
\begin{gathered}
(\op{\vec{l}} \times \op{\vr})_i = \epsilon_{ijk}\epsilon_{jab} \op{r}_a \op{p}_b \op{r}_k = - \epsilon_{ikj}\epsilon_{abj} \op{r}_a \op{p}_b \op{r}_k = \\ = 
- (\delta_{ia}\delta_{kb} - \delta_{ib}\delta_{ka}) \op{r}_a \op{p}_b \op{r}_k = - \op{r}_i \op{p}_k \op{r}_k + \op{r}_k \op{p}_i \op{r}_k =
- \op{r}_i (\op{\vp} \cdot \op{\vr}) + r^2 \op{p}_i - \mathbf{i} \op{r}_i
\end{gathered}
\end{equation}\\

Распишем искомый коммутатор:

$$
\begin{gathered}
{[\op{H},\op{\vec{A}}]} =  \brs{ \brc{\frac{\op{\vp}^2}{2}-\frac{1}{r}} ,~  \brc{\frac{\op{\vr}}{r} + \frac{1}{2} \brc{\op{\vec{l}} \times \op{\vp}} - \frac{1}{2} \brc{ \op{\vp} \times \op{\vec{l}}}} } = \\ =
\brs{\frac{\op{\vp}^2}{2}, \frac{1}{2} \brc{\op{\vec{l}} \times \op{\vp}}} - 
\brs{\frac{\op{\vp}^2}{2}, \frac{1}{2} \brc{\op{\vp} \times \op{\vec{l}}}} +
\brs{-\frac{1}{r}, \frac{\op{\vr}}{r}} + \\ +
\brs{-\frac{1}{r}, \frac{1}{2} \brc{\op{\vec{l}} \times \op{\vp} - \op{\vp} \times \op{\vec{l}}}} +
\brs{\frac{\op{\vp}^2}{2}, \frac{\op{\vr}}{r}}
\end{gathered}
$$\\

Вычислим каждый из коммутаторов отдельно:
\begin{equation}
\label{proof:1}
\begin{gathered}
\brs{\frac{\op{\vp}^2}{2}, \frac{1}{2} \brc{\op{\vec{l}} \times \op{\vp}}} =
-\frac{1}{4} \brs{\brc{\op{\vec{l}} \times \op{\vp}}, \op{\vp}^2} = 
\left. -\frac{1}{4} \epsilon_{ijk} \brs{\op{l}_j \op{p}_k, \op{p}_m^2 } \right|_{\text{\eqref{eq:eq2}}} = \\ =
 -\frac{1}{4} \epsilon_{ijk} \op{l}_j \brs{ \op{p}_k, \op{p}_m^2 }  - \frac{1}{4} \epsilon_{ijk} \underbrace{ \brs{\op{l}_j , \op{p}_m^2 } }_{=0} \op{p}_k = \\ =
 -\frac{1}{4} \epsilon_{ijk} \op{l}_j \brc{  \op{p}_r \brs{ \op{p}_k, \op{p}_m } - \brs{ \op{p}_k, \op{p}_m } \op{p}_r } = 0
\end{gathered}
\end{equation}\\

\begin{equation}
\label{proof:2}
\brs{\frac{\op{\vp}^2}{2}, \frac{1}{2} \brc{\op{\vp} \times \op{\vec{l}}}} = 0 ~~ \text{(доказывается аналогично)}
\end{equation}\\

\begin{equation}
\label{proof:3}
\begin{gathered}
\brs{-\frac{1}{r}, \frac{\op{r}_i}{r}} = 
\brs{-\frac{1}{\sqrt{r_k r^k}}, \frac{\op{r}_i}{\sqrt{r_m r^m}}} = \\ =
- \frac{\op{r}_i}{r_k r^k} + \frac{\op{r}_i}{\sqrt{r_m r^m}} \frac{1}{\sqrt{r_k r^k}} =
- \frac{\op{r}_i}{r_k r^k} + \frac{\op{r}_i}{r_k r^k} = 0
\end{gathered}
\end{equation}

\begin{equation}
\label{proof:4}
\begin{gathered}
-\brs{\frac{1}{r}, \brc{\op{\vec{l}} \times \op{\vp} - \op{\vp} \times \op{\vec{l}}}} = 
-\epsilon_{ijk} \brs{\frac{1}{r}, \op{l}_i \op{p}_j + \op{p}_j \op{l}_i} = \\ =
-\epsilon_{ijk} \brc{ \brs{\frac{1}{r}, \op{l}_i \op{p}_j} + \brs{\frac{1}{r}, \op{p}_j \op{l}_i} } = \\ = 
-\epsilon_{ijk} \brc{ \op{l}_i \brs{\frac{1}{r}, \op{p}_j} + \underbrace{ \brs{\frac{1}{r}, \op{l}_i} }_{=0} \op{p}_j + \op{p}_j \underbrace{ \brs{\frac{1}{r}, \op{l}_i} }_{=0} +  \brs{\frac{1}{r}, \op{p}_j } \op{l}_i} = \\ =
\left. - \epsilon_{ijk} \brc{ \op{l}_i \brs{\frac{1}{r}, \op{p}_j} + \brs{\frac{1}{r}, \op{p}_j } \op{l}_i} \right|_{\text{\eqref{eq:eq5}}} = 
\epsilon_{ijk} \brc{\op{l}_i \frac{\mathbf{i} \op{r}_j}{r^3} + \frac{\mathbf{i} \op{r}_j}{r^3} \op{l}_i } = \\ = 
\left. \frac{\mathbf{i}}{r^3} \brc{ (\op{\vec{l}} \times \op{\vr}) -(\op{\vr} \times \op{\vec{l}}) } \right|_{\text{\eqref{eq:eq7}, \eqref{eq:eq8}}} = 
\frac{\mathbf{i}}{r^3} \brc{ 2 r^2 \op{p}_i - \op{r}_i (\op{\vp} \cdot \op{\vr}) - \mathbf{i} \op{r}_i - \op{r}_i (\op{\vr} \cdot \op{\vp}) }
\end{gathered}
\end{equation}

\begin{equation}
\label{proof:5}
\begin{gathered}
\brs{\op{\vp}^2, \frac{\op{r}_j}{r}} =\left. \op{p}_i \brs{\op{p}_i, \frac{\op{r}_j}{r}} + \brs{\op{p}_i, \frac{\op{r}_j}{r}} \op{p}_i \right|_{\text{\eqref{eq:eq6}}} = \\ =
\mathbf{i}\,\op{p}_i \brc{\frac{\op{r}_i \op{r}_j}{r^3} - \frac{\delta_{ij}}{r}} + \mathbf{i}\brc{\frac{\op{r}_i \op{r}_j}{r^3} - \frac{\delta_{ij}}{r}} \op{p}_i = \\ = 
\mathbf{i} \brc{ \op{p}_i \frac{\op{r}_i \op{r}_j}{r^3} - \op{p}_j \frac{1}{r} + \frac{\op{r}_i \op{r}_j}{r^3} \op{p}_i - \frac{1}{r} \op{p}_j } = \\ = 
\mathbf{i}\brc{ (\op{\vp} \cdot \op{\vr}) \frac{\op{r}_j}{r^3} + \mathbf{i} \frac{\op{r}_i}{r^3} - \frac{1}{r} \op{p_j} + \frac{\op{r}_j}{r^3} (\op{\vr} \cdot \op{\vp}) - \frac{1}{r} \op{p_j} } = \\ =
\frac{\mathbf{i}}{r^3} \brc{ - 2 r^2 \op{p}_i + \op{r}_i (\op{\vp} \cdot \op{\vr}) + \mathbf{i} \op{r}_i + \op{r}_i (\op{\vr} \cdot \op{\vp}) }
\end{gathered}
\end{equation}\\

Следовательно:
$$
\begin{aligned}
{[\op{H},\op{\vec{A}}]} = 
\frac{\mathbf{i}}{2 r^3} \brc{ 2 r^2 \op{p}_i - \op{r}_i (\op{\vp} \cdot \op{\vr}) - \mathbf{i} \op{r}_i - \op{r}_i (\op{\vr} \cdot \op{\vp}) } + \\
+ \frac{\mathbf{i}}{2 r^3} \brc{ -2 r^2 \op{p}_i + \op{r}_i (\op{\vp} \cdot \op{\vr}) + \mathbf{i} \op{r}_i + \op{r}_i (\op{\vr} \cdot \op{\vp}) } = 0
\end{aligned}
$$


\end{proof}