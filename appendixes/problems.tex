\appendix

\renewcommand{\thesection}{}
\renewcommand{\thesubsection}{}
\renewcommand{\thesubsubsection}{}

\setcounter{chapter}{0}
\setcounter{section}{0}

\chapter{Задания первого семестра}

\section{Задание 1}

\subsection*{\hfil Упражнения \hfil}

\begin{exercise}
\item\label{ex:1_1_1}%
Найти операторы эрмитово сопряжённые и обратные по отношению к операторам: (а) инверсии $\op{I}$ и (б) трансляции $\op{T}_a$.
%
\item\label{ex:1_1_2}%
Доказать, что операторы координаты, импульса и энергии эрмитовы.
%
\item\label{ex:1_1_3}%
Доказать, что оператор $\op{A}^\dag \op{A}$ является эрмитовым (при любом $\op{A}$) и $\bfk{\psi}{\op{A}^\dag \op{A}}{\psi} \ge 0$ (при любом $\psi$).
%
\item\label{ex:1_1_4}%
Найти собственные значения и собственные функции оператора инверсии $\op{I}$.
%
\item\label{ex:1_1_5}%
Найти собственные значения и собственные функции оператора трансляции $\op{T}_a$.
%
\item\label{ex:1_1_6}%
Убедитесь в справедливости следующих соотношений:
$$
\begin{gathered}
{[\op{A} \op{B}, \op{C}]} = \op{A} [\op{B},\op{C}] + [\op{A},\op{C}]\op{B} \\
{[\op{A}, \op{B} \op{C}]} = \op{B}[\op{A},\op{C}] + [\op{A},\op{B}]\op{C}
\end{gathered}
$$
%
\item\label{ex:1_1_7}%
Раскройте следующие коммутаторы:
$$
\brs{\op{x}, \op{p}^2_x}, ~~~
\brs{U(x), \op{p}_x} ~~~
\brs{\op{T}_a, \op{x}} ~~~
\brs{\op{T}_a, \op{p}_x}
$$
%
\item\label{ex:1_1_8}%
Найти явный вид оператора $e^{i\phi\op{I}}$.
%
\item\label{ex:1_1_9}%
Получить разложение
$$
e^{\xi \op{A}} \op{B} e^{-\xi \op{A}}
  = \op{B} + \xi \brs{\op{A}, \op{B}}
    + \frac{1}{2!} \xi^2 \brc{\op{A}, \brc{\op{A}, \op{B}}} + \dots
$$
%
\item\label{ex:1_1_10}%
Упростить выражение
$$
e^{i\vec{a}\op{\vp}/\hbar} U(\vr) e^{-i\vec{a}\op{\vp}/\hbar}
$$%
%
где $\vec{a}$ -- постоянный вектор.
\end{exercise}

\subsection*{\hfil Задачи \hfil}

\begin{problem}
\item\label{p:1_1_1}%
Частица массы $m$ движется в одномерном потенциальном <<ящике>> с бесконечно высокими стенками:
$$
U(x) = \begin{cases}
0, & 0 < x < a\\
+\infty, & x < 0,~ x > a
\end{cases}
$$%
%
Найти уровни энергии $E_n$ и волновые функции $\psi_n(x)$ стационарных состояний. Вычислить средние значения $\avg{x}$, $\avg{p}$, $\avg{(\Delta x)^2}$ и $\avg{(\Delta p)^2}$ для $n$-го стационарного состояния. Обсудить величину $\avg{(\Delta x)^2}\avg{(\Delta p)^2}$ в связи с соотношением неопределённостей.\\
Нарисуйте траектории движения классической частицы в фазовом пространстве $(x, p)$, отвечающие энергиям $E_n$. Исследуйте, как меняется площадь в фазовом пространстве, охватываемая каждой такой траекторией при переходе из одного стационарного состояния к другому (соответствующее приращение площади называют <<фазовым объектом, приходящимся на одно квантовое состояние>>).
%
\item\label{p:1_1_2}%
Частица массы $m$ совершает финитное движение в одномерной <<прямоугольной>> потенциальной яме конечной глубины:
$$
U(x) =
\begin{cases}
-U_0, & \abs{x} < a,\\
0, & \abs{x} > a.
\end{cases}
$$%
%
Найти уровни энергии $E_n$ и волновые функции $\psi_n(x)$ стационарных состояний. Исследуйте, существуют ли связанные состояния в <<прямоугольной>> потенциальной яме фиксированной ширины $2a$, если $U_0 \to 0$. Согласуется ли результат с соотношением неопределенностей?\\
%
Воспользуйтесь полученными результатами для оценки числа уровней электрона в металлическом образце (глубина потенциала $U_0 = 10~\text{эВ}$ примерно равна работе выхода), если (а) $a = 0.1~\text{нм}$ (<<атом>>); (б) $a = 10~\text{нм}$ (<<наночастица>>); (в) $a = 1~\text{см}$ (<<макроскопический образец>>).
%
\item\label{p:1_1_3}%
Найти уровни энергии и волновые функции стационарных состояний для частицы массы $m$ в одномерной потенциальной яме следующего вида:
$$
U(x) =
\begin{cases}
+\infty, & x < 0\\
-U_0, & 0 < x < a,\\
0, & x > a
\end{cases}
$$%
%
Что здесь можно сказать о связанных состояниях при фиксированном $a$ и $U_0 \to 0$?
%
\item\label{p:1_1_4}%
Частица массы $m$ совершает финитное движение в одномерной модельной потенциальной яме, вид которой может быть представлен $\delta$-функцией:
$$
U(x) = -\frac{\hbar^2 \kappa_0}{m} \delta(x)
$$%
%
где $\kappa_0$ -- параметр ямы. Покажите, что в этой яме имеется только одно связанное состояние; найдите энергию уровня и волноую функцию частицы в координатном представлении. Вычислить $\avg{x}$, $\avg{p}$, $\avg{(\Delta x)^2}$ и $\avg{(\Delta p)^2}$ в этом состоянии.
%
\item\label{p:1_1_5}%
Частица массы $m$ находится в связанном состоянии в $\delta$-потенциале (см.~\cref{p:1_1_4}). В момент $t=0$ происходит мгновенное изменение параметра ямы от $\kappa_0$ до $\kappa_1$. Найти вероятность <<ионизации>>. Обсудить эволюцию волновой функции частицы сразу после ионизации в случае, когда $\kappa_1 = 0$.
%
\item\label{p:1_1_6}%
Частица массы $m$ движется в одномерном потенциальном поле вида
$$
U(x) =
\begin{cases}
+\infty, & x < 0,\\
-A\delta(x-a), & x > 0.
\end{cases}
$$%
%
Найти зависимость числа связанных состояний от параметров $a$ и $A$.
%
\item\label{p:1_1_7}%
Частица массы $m$ совершает финитное движение в одномерном потенциальном поле вида
$$
U(x) = -\frac{\hbar^2 \kappa_0}{m} \brc{\delta(x+a) + \delta(x-a)}
$$%
%
где $\kappa_0$ -- параметр потенциала. Найти энергии уровней и волновые функции стационарных состояний. Как зависит число связанных состояний от параметров $a$ и $\kappa_0$\\
Рассматривая эту задачу как модель молекулярного иона водорода $H_2$, исследуйте зависимость энергий уровней от $a$ при фиксированном $\kappa_0$.
%
\item\label{p:1_1_8}%
Покажите, что в случае $\kappa_0 a \gg 1$ в \cref{p:1_1_7} связанные состояние представляют собой дуплет близко расположенных уровней. Обсудить связь со структурой низколежищих уровняй молекулы аммиака $NH_3$. В этом же пределе $\kappa_0 a \gg 1$ найдите вероятность нахождения частицы в момент $t$ в правой яме ($x = a$), если при $t=0$ они находилась в левой яме ($x = -a$).
%
\item\label{p:1_1_9}%
Частица массы $m$ движется в одномерном потенциальном поле вида
$$
U(x) =
\begin{cases}
A\delta(x), & \abs{x} < a,\\
+\infty, & \abs{x} > a.
\end{cases}
$$%
где $A > 0$. Найти энергии уровней и волновые функции стационарных состояний. В случае, когда $maA/\hbar^2 \gg 1$, исследуйте положения уровней в нижней части спектра.
%
\item\label{p:1_1_10}%
Частица массы $m$ движется в одномерном потенциальном поле вида
$$
U(x) = -\frac{\hbar^2 \kappa_0}{m} \sum_{n=-\infty}^{n=+\infty} \delta(x - na)
$$%
%
где $\kappa_0$ и $a$ -- параметры потенциала. Исследуйте, при каких отрицательных и положительных энергиях $E$ частицы такое движение возможно. Покажите, что имеются зоны <<разрешённых>> и <<запрещённых>> энергий.\\
Исследуйте, что происходит с ширинами зон в предельных случаях $\kappa_0 a \gg 1$ (сильная связь) и $\kappa_0 a \ll 1$ (слабая связь).
%
\item\label{p:1_1_11}%
Пусть в некоторый фиксированный момент $t$ состояние частицы массы $m$ описывается волновой функцией $\psi_0(x)$ такой, что $\avg{x} = x_0$, $\avg{p} = p_0$, а произведение неопределённостей координаты и импульса этой частицы принимает минимальное значение, т.е.
$$
\avg{(\op{x} - x_0)^2} \avg{(\op{p} - p_0)^2} = \frac{\hbar^2}{4}
$$%
%
Состояние частицы, которое описывается такой волновой функцией $\psi_0(x)$ называется <<когерентным>>. Докажите, что
$$
\psi_0(x) =
  \frac{1}{(2\pi\sigma^{2}_{x})^{1/4}}
  e^{i\frac{p_0 x}{\hbar} - \frac{(x - x_0)^2}{4\sigma^{2}_{x}}}, ~~~
  \sigma^2_x \equiv \avg{(\op{x} - x_0)^2}
$$%
%
Исследуйте, как меняется во времени волновая функция свободной частицы $\Psi(x, t)$ (и $\abs{\Psi(x, t)}^2$), если $\Psi(x, 0) = \psi_0(x)$.
%
\item\label{p:1_1_12}%
Частица массы $m$ свободно движения вдоль оси $x$ с энергией $E$ и в области $x > 0$ попадает в область действия потенциала, который имеет вид: (а) прямоугольной потенциальной ямы ширины $a$ и глубины $U_0$, (б) прямоугольного потенциального барьера ширины $a$ и высоты $U_0$. Найдите коэффициенты прохождения $T(E)$ и отражения $R(E)$ частицы от указанных потенциалов; нарисуйте графики. Существуют ли энергии, при которых ямы и барьеры полностью прозрачны для падающих частиц? Если <<да>>, то сформулируйте, чем определяются эти энергии.\\
В случае потенциальной ямы предложите способ оценки энергии $E_0$, выше которой квантовые ответы практически совпадают с классическими. Какова эта энергия при прохождении электрона сквозь наноскопический слой металла: $a = 10~\text{нм}$ и $U_0 = 10~\text{эВ}$?
%
\item\label{p:1_1_13}%
Частица массы $m$ свободно движется вдоль оси $x$ с энергией $E$ и попадает в область действия $\delta$-потенциала (см.~\cref{p:1_1_4}). Найдите коэффициенты прохождения $T(E)$ и отражения $R(E)$ частицы; нарисуйте графики.
\end{problem}


\section{Задание 2}

\subsection*{\hfil Упражнения \hfil}
%
\begin{exercise}
\item\label{ex:1_2_1}%
Как выглядят в импульсном представлении операторы координаты и импульса?
%
\item\label{ex:1_2_2}%
Воспользовавшись операторами понижения и повышения, $\ann$ и $\cre$, найти средние значения операторов $\op{x}^2$, $\op{x}^4$ и $\op{x}^{2k+1}$, а также $\op{p}^2$, $\op{p}^4$ и $\op{p}^{2k+1}$ в $n$-м стационарном состоянии линейного гармонического осциллятора. Обсудить величину $\avg{x^2}\avg{p^2}$ в связи с соотношением неопределённостей.
%
\item\label{ex:1_2_3}%
Найти явный вид операторов понижения и повышения, $\op{a}(t)$ и $\op{a}^\dag(t)$ для линейного гармонического осциллятора в представлении Гайзерберга.
%
\item\label{ex:1_2_4}%
Доказать эрмитовость оператора орбитального момента.
%
\item\label{ex:1_2_5}%
Раскройте следующие коммутаторы:
$$
\begin{gathered}
\brs{\op{l}_i, x_j},~~ \brs{\op{l}_i, \op{p}_j},~~
\brs{\op{l}_i, \op{l}_j},~~ \brs{\op{l}_i, \op{\vp}^2},~~
\brs{\op{l}_i, (\vr \op{\vp})},~~ \\
%
\brs{\op{l}_i, U(r)},~~
\brs{\op{l}_z, U(\rho)}~ (\rho = \sqrt{x^2 + y^2}),~~ \brs{\op{l}_i, \vec{l}^2}
\end{gathered}
$$
%
\item\label{ex:1_2_6}%
Воспользовавшись явным видом матриц Паули, доказать справедливость следующих соотношений:
$$
\begin{gathered}
\sigma_k \sigma_l = \delta_{kl} + ie_{klm} \sigma_m \\
(\vec{\sigma} \vec{A}) (\vec{\sigma} \vec{B})
  = (\vec{A} \vec{B}) + i(\vec{\sigma} \brs{\vec{A} \times \vec{B}})
\end{gathered}
$$%
%
где $\vec{A}$ и $\vec{B}$ -- произвольные векторы.
%
\item\label{ex:1_2_7}%
Найти явный вид оператора $e^{i\alpha(\vec{\sigma} \vec{n})}$
%
\item\label{ex:1_2_8}%
Построить матрицы операторов углового момента $\op{j}_x$, $\op{j}_y$, $\op{j}_z$, а также $\opv{j}^2$, $\op{j}_+$ и $\op{j}_-$ для квантовой системы с угловым моментом $j = 1$. Как выглядят собственные векторы операторов $\opv{j}^2$ и $\op{j}_z$?
%
\item\label{ex:1_2_9}%
Найти собственные значения и собственные векторы спинового оператора $\op{j}_n = (\opv{j}\vec{n})$ для квантовой системы с угловым моментом $j=1$.
%
\item\label{ex:1_2_10}%
Эрмитовый оператор с дискретным спектром $\op{f}(\lambda)$ зависит от параметра $\lambda$. Соответственно собственные значения $f_n(\lambda$, и собственные векторы $\ket{n(\lambda)}$ этого оператора также зависят от $\lambda$. Доказать следующее соотношение:
$$
\pd{f_n(\lambda)}{\lambda} =
  \bfkh{n}{\pd{\op{f}(\lambda)}{\lambda}}{n}
$$
%
\item\label{ex:1_2_11}%
Потенциальная энергия взаимодействия двух частиц с массами $m_1$ и $m_2$, $U(\abs{\vr_1 - \vr_2})$, зависит только от расстояния между частицами. Запишите стационарное уравнение Шрёдингера, определяющее волновую функцию состояния этих частиц с определённой энергией $E$. Покажите, что для решения этого уравнения удобно воспользоваться следующими переменными:
$$
\vec{R} = \frac{m_1 \vr_1 + m_2 \vr_2}{m_1 + m_2},~~~
\vr = \vr_1 - \vr_2
$$%
%
Какой вид принимает при этом волновая функция $\Psi(\vr_1, \vr_2)$?
%
\item\label{ex:1_2_12}%
Релятивистские поправки к энергии стационарного состояния $\ket{nlm}$ атома водорода определяются средними значениями операторов $\op{\vp}^4$ и $1/r^3$. Найти эти средние значения, а также средние значения операторов $1/r$ и $1/r^2$ в состоянии $\ket{nlm}$.
\end{exercise}

\subsection*{\hfil Задачи \hfil}
%
\begin{problem}
\item ...
\end{problem}



\chapter{Задания второго семестра}

\section{Задание 1}

\subsection*{\hfil Упражнения \hfil}
%
\begin{exercise}
\item ...
\end{exercise}

\subsection*{\hfil Задачи \hfil}
%
\begin{problem}
\item ...
\end{problem}
