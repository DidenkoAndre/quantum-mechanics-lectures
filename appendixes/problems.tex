\appendix

\renewcommand{\thesection}{}
\renewcommand{\thesubsection}{}
\renewcommand{\thesubsubsection}{}

\setcounter{chapter}{0}
\setcounter{section}{0}

\chapter{Задания первого семестра}

\section{Задание 1}

\subsection*{\hfil Упражнения \hfil}

\begin{exercise}
\item\label{ex:1_1_1}%
Найти операторы эрмитово сопряжённые и обратные по отношению к операторам: (а) инверсии $\op{I}$ и (б) трансляции $\op{T}_a$.
%
\item\label{ex:1_1_2}%
Доказать, что операторы координаты, импульса и энергии эрмитовы.
%
\item\label{ex:1_1_3}%
Доказать, что оператор $\op{A}^\dag \op{A}$ является эрмитовым (при любом $\op{A}$) и $\bfk{\psi}{\op{A}^\dag \op{A}}{\psi} \ge 0$ (при любом $\psi$).
%
\item\label{ex:1_1_4}%
Найти собственные значения и собственные функции оператора инверсии $\op{I}$.
%
\item\label{ex:1_1_5}%
Найти собственные значения и собственные функции оператора трансляции $\op{T}_a$.
%
\item\label{ex:1_1_6}%
Убедитесь в справедливости следующих соотношений:
$$
\begin{gathered}
{[\op{A} \op{B}, \op{C}]} = \op{A} [\op{B},\op{C}] + [\op{A},\op{C}]\op{B} \\
{[\op{A}, \op{B} \op{C}]} = \op{B}[\op{A},\op{C}] + [\op{A},\op{B}]\op{C}
\end{gathered}
$$
%
\item\label{ex:1_1_7}%
Раскройте следующие коммутаторы:
$$
\brs{\op{x}, \op{p}^2_x}, ~~~
\brs{U(x), \op{p}_x} ~~~
\brs{\op{T}_a, \op{x}} ~~~
\brs{\op{T}_a, \op{p}_x}
$$
%
\item\label{ex:1_1_8}%
Найти явный вид оператора $e^{i\phi\op{I}}$.
%
\item\label{ex:1_1_9}%
Получить разложение
$$
e^{\xi \op{A}} \op{B} e^{-\xi \op{A}}
  = \op{B} + \xi \brs{\op{A}, \op{B}}
    + \frac{1}{2!} \xi^2 \brc{\op{A}, \brc{\op{A}, \op{B}}} + \dots
$$
%
\item\label{ex:1_1_10}%
Упростить выражение
$$
e^{i\vec{a}\op{\vp}/\hbar} U(\vr) e^{-i\vec{a}\op{\vp}/\hbar}
$$%
%
где $\vec{a}$ -- постоянный вектор.
\end{exercise}

\subsection*{\hfil Задачи \hfil}

\begin{problem}
\item\label{p:1_1_1}%
Частица массы $m$ движется в одномерном потенциальном <<ящике>> с бесконечно высокими стенками:
$$
U(x) = \begin{cases}
0, & 0 < x < a\\
+\infty, & x < 0,~ x > a
\end{cases}
$$%
%
Найти уровни энергии $E_n$ и волновые функции $\psi_n(x)$ стационарных состояний. Вычислить средние значения $\avg{x}$, $\avg{p}$, $\avg{(\Delta x)^2}$ и $\avg{(\Delta p)^2}$ для $n$-го стационарного состояния. Обсудить величину $\avg{(\Delta x)^2}\avg{(\Delta p)^2}$ в связи с соотношением неопределённостей.\\
Нарисуйте траектории движения классической частицы в фазовом пространстве $(x, p)$, отвечающие энергиям $E_n$. Исследуйте, как меняется площадь в фазовом пространстве, охватываемая каждой такой траекторией при переходе из одного стационарного состояния к другому (соответствующее приращение площади называют <<фазовым объектом, приходящимся на одно квантовое состояние>>).
%
\item ...
\end{problem}


\section{Задание 2}

\subsection*{\hfil Упражнения \hfil}
%
\begin{exercise}
\item\label{ex:1_2_1}%
Как выглядят в импульсном представлении операторы координаты и импульса?
%
\item\label{ex:1_2_2}%
Воспользовавшись операторами понижения и повышения, $\ann$ и $\cre$, найти средние значения операторов $\op{x}^2$, $\op{x}^4$ и $\op{x}^{2k+1}$, а также $\op{p}^2$, $\op{p}^4$ и $\op{p}^{2k+1}$ в $n$-м стационарном состоянии линейного гармонического осциллятора. Обсудить величину $\avg{x^2}\avg{p^2}$ в связи с соотношением неопределённостей.
%
\item\label{ex:1_2_3}%
Найти явный вид операторов понижения и повышения, $\op{a}(t)$ и $\op{a}^\dag(t)$ для линейного гармонического осциллятора в представлении Гайзерберга.
%
\item\label{ex:1_2_4}%
Доказать эрмитовость оператора орбитального момента.
%
\item\label{ex:1_2_5}%
Раскройте следующие коммутаторы:
$$
\begin{gathered}
\brs{\op{l}_i, x_j},~~ \brs{\op{l}_i, \op{p}_j},~~
\brs{\op{l}_i, \op{l}_j},~~ \brs{\op{l}_i, \op{\vp}^2},~~
\brs{\op{l}_i, (\vr \op{\vp})},~~ \\
%
\brs{\op{l}_i, U(r)},~~
\brs{\op{l}_z, U(\rho)}~ (\rho = \sqrt{x^2 + y^2}),~~ \brs{\op{l}_i, \vec{l}^2}
\end{gathered}
$$
%
\item\label{ex:1_2_6}%
Воспользовавшись явным видом матриц Паули, доказать справедливость следующих соотношений:
$$
\begin{gathered}
\sigma_k \sigma_l = \delta_{kl} + ie_{klm} \sigma_m \\
(\vec{\sigma} \vec{A}) (\vec{\sigma} \vec{B})
  = (\vec{A} \vec{B}) + i(\vec{\sigma} \brs{\vec{A} \times \vec{B}})
\end{gathered}
$$%
%
где $\vec{A}$ и $\vec{B}$ -- произвольные векторы.
%
\item\label{ex:1_2_7}%
Найти явный вид оператора $e^{i\alpha(\vec{\sigma} \vec{n})}$
\end{exercise}

\subsection*{\hfil Задачи \hfil}
%
\begin{problem}
\item ...
\end{problem}



\chapter{Задания второго семестра}

\section{Задание 1}

\subsection*{\hfil Упражнения \hfil}
%
\begin{exercise}
\item ...
\end{exercise}

\subsection*{\hfil Задачи \hfil}
%
\begin{problem}
\item ...
\end{problem}